\section{Recapitulation}\label{sec:ch7-recap}

In this thesis, I have used the mouse tracking paradigm
across a series of experiments to investigate conflict in reasoning.
In doing so, I hoped to further understand
under what circumstances conflict arises,
what form this conflict takes,
and at what points in time it occurs.

In Experiments 1 and 2 (Chapter 3),
I pitted perceptual cues in the form of visual similarity,
against conceptual knowledge in the form of category membership,
in a forced-choice induction task,
using both natural (Experiment 1) and artificial categories (Experiment 2).
In both experiments, I found that
perceptual cues are an early driver of participants' motor output,
and that conceptual knowledge is brought online later in reasoning,
sometimes causing participants to change direction
if they had begun to move towards the perceptually-cued option.
Analysis of the time course data showed that
participants began to be drawn towards
the perceptually-cued foil from 300 msec (in both experiments),
and did not begin to inhibit this attraction until \tildetext750--1,500 msec.
In Experiment 2, I further showed that participants were more likely
to override their perceptually-driven movements
when reasoning about properties that were conceptually related
to the distinction between the categories,
likely because these properties make conceptual knowledge more accessible,
or help participants realise that the perceptual cues were inappropriate.
Again, this effect, being dependent on conceptual knowledge,
was not visible until \tildetext620 msec.

Experiments 3 and 4 (Chapter 4) were similar,
but pitted associative knowledge against structured knowledge.
In Experiment 3, I contrasted conflict trials,
where the foil response was strongly associated with the base
according to separate association ratings,
to control trials, where it was not.
In Experiment 4, I collected association ratings
from each participant for each pair of species.
This allowed me to conduct regression analyses
with both each participants' associative knowledge
(the ratio of the association ratings in favour of each response)
and structured knowledge as predictors.

In both experiments, I again found that participants were
influenced by both kinds of information:
in Experiment 3 participants gave the correct response
on \tildetext95\% of control trials but only \tildetext75\% of conflict trials,
while in Experiment 4 both structured and associative knowledge
were significant predictors of participants' responses.
However, compared to the previous experiments,
there was less evidence here that participants were
initially drawn towards the associatively-cued option,
and then subsequently towards the structured response.
While this did happen on some trials,
on most trials participants either moved straight to one response option,
or straight to the other.

In Experiments 5 and 6,
I explored conflict from the perspective of
dual process theories of reasoning.
In Experiment 5 (Chapter 5),
participants completed a mouse tracking version of
the base rate neglect task \citep{Kahneman1973,DeNeys2008}
where they could draw on either stereotypical descriptions
or statistical base rates to decide someone's social category.
I manipulated both the descriptions and the base rates,
so that they could either agree, disagree,
or only one cue was informative.

Consistent with a default-interventionist dual process model \citep[i.e.][]{Evans2006},
I found that participants' responses were predominantly determined by
the contents of the descriptions,
and that participants who opted to ignore the description
and rely on the base rate instead experienced conflict.
In line with previous work \citep[e.g.][]{Tversky1982,Kahneman1973},
but counter to intuitive logic accounts \citep{DeNeys2008,Pennycook2014}
or a parallel-competitive dual-process account \citep{Sloman1996},
the influence of the base rates was less pronounced.
Specifically, participants did sometimes give the base rate-cued response
even when it conflicted with the description (\tildetext20\% of trials).
Aside from this, however,
participants responding on the basis of the descriptions
did not show signs of conflict when the base rate was manipulated
to disagree with the description.
A Bayesian follow-up analysis, finally,
showed that despite the non-significant effect,
there was not considerable evidence for the existence of a null effect either.
Rather, it seems that participants were
very slightly (\tildetext 3\%, or around 30 msec) slower
giving the description-cued response
when it disagreed with the base rate.

In Experiment 6 (Chapter 6),
participants completed the more complicated
Cognitive Reflection Test \citep[CRT;][]{Frederick2005}:
a series of questions for which
the first response that comes to mind is often incorrect.
Unlike Experiments 1 to 5,
there were four response options here,
located in each corner of the screen.
Participants were allowed to respond in their own time,
and I analysed movements of the mouse cursor
over the first sixty seconds of each trial
as  participants moved the mouse around the screen deciding on a response.
This novel form of mouse tracking
made it possible to record participants' ongoing decisions over a longer period of time.
It also allowed me to infer participants' attraction towards four response options,
rather than the two used in standard mouse tracking.
This was useful in differentiating between
effects where participants are actually drawn
towards a particular alternative response option,
and those where participants are merely slow
to move towards the option they do choose.
I also included, following \citet{DeNeys2013a},
no-conflict versions of the CRT problems
to serve as a control condition where
participants' first, heuristic responses are the correct ones.
Finally, this experiment was, to my knowledge,
the first to record response latencies on any version of the CRT.

Consistent with a dual process perspective,
heuristic responses on the CRT were given more quickly,
and participants were faster to approach these response options before selecting them,
than correct options.
Contrary to response time analyses of other tasks \citep[e.g.][]{DeNeys2008}
and confidence rating data from the CRT \citep{DeNeys2013a,Gangemi2015} however,
participants were no faster to give
the heuristic (and correct) response on no-conflict problems
than to give the heuristic (and incorrect) response on conflict problems.

The nature of the mouse tracking data here allowed for a number of novel analyses.
First, I demonstrated that on conflict problems
participants began to be drawn towards the heuristic option from 3.7 seconds,
and towards the correct option from 9.3 seconds,
including time spent reading the problems.
More relevantly, to test for the presence of conflict
I calculated, for trials where participants gave one response,
the degree to which their mouse cursor was in the region of
the alternative, competitor response beforehand,
compared to the regions of the other two non-competing responses.
Thus, for instance, if a participant giving the correct response
is drawn towards the heuristic option while doing so,
their cursor will spend more time in the region of the heuristic option,
on average, than either of the other two foil options.
Again in line with a default-interventionist model,
I found exactly this effect,
as participants were drawn towards the heuristic option
before selecting the correct one.
However, contrary to some other accounts, the reverse was not true:
participants did not hover over the correct option
before ultimately selecting the heuristic one.

In this final chapter,
I try to make sense of these results
and discuss the implications they have for what we know about conflict in reasoning.
I do this both within the specific sub-domains
that these experiments addressed ---
information selection in induction,
and dual process theories of reasoning ---
and in terms of conflict in reasoning and cognition more broadly.

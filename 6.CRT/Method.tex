

\section{Method}

\subsection{Participants}

One hundred and thirty one students at Queen’s University Belfast
participated in exchange for course credit.

\subsection{Materials}

Eight problems were adapted from
\citegap{Primi2015}{'s} extended version of the CRT.
Each of these problems was modified to create a set of
eight corresponding non-conflict problems,
in which the intuitively appealing responses were also the correct ones
(see the Appendix~\ref{appendix:exp6_stimuli}).
Participants were randomly allocated to complete either
conflict versions of items 1, 3, 5, and 7
and no-conflict versions of the rest, or vice versa.
Each problem was presented in a 4-option multiple choice format.
For the conflict items, the possible responses were the correct option,
the incorrect heuristic option, and two incorrect foil options.
For the non-conflict items the correct intuitive option was presented
with three incorrect foils.

\subsection{Procedure}

The experiment was administered on personal computers,
programmed using PsychScript (see Chapter 2),
and run in the web browser.
Participants were instructed to respond in their own time to each CRT problem
by clicking on one of the four response options
presented in the four corners of the display
(Figure~\ref{fig:exp6-screenshot}).
Participants were not made aware of
the mouse tracking aspect of the experiment in advance. 

\begin{figure}[ht]
  \centering
  \includegraphics[width=\figurewidth]{imgs/exp6-screenshot.png}
  \caption[A screen shot from the CRT, Experiment 6.]{
    \label{fig:exp6-screenshot}
    A screen shot from the CRT.
  }
\end{figure}

Each problem was preceded by onscreen instructions to
click on a button marked ``Go'', presented in the centre of the monitor.
This was done to ensure the mouse cursor was located
in the same central position at the beginning of each trial.
The button was then replaced by the problem text
and the four response options appeared simultaneously in the corners
(Figure~\ref{fig:exp6-screenshot}).
The response options were randomly assigned to the four locations on each trial,
with the constraint that the correct and heuristic response options
were always adjacent for conflict problems.
The mouse cursor was no longer visible at the onset of each trial
to prevent it from obscuring the question text.
The cursor reappeared once it had been moved more than 5\% of the width of the display.
Mouse cursor location was recorded every 20 msec.

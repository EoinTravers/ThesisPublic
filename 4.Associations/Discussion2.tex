
\subsection{Discussion}

In general, the results of this experiment
are consistent with Experiment 3,
but unlike in Experiment 3, I did not exclude any data
on the basis of the post-test results.
Participants' inferences were influenced by both
associative and structured knowledge.
Associative knowledge was indexed by the association ratio,
reflecting how strongly each participant rated
the association between the base and the correct species,
compared to the base and the foil.
This was a strong predictor of their inferences,
with participants more likely to project the property
to the correct species as its association with the base increased,
relative to the foil.
If participants only relied on associative knowledge, however,
they should give the correct response only 50\% of the time
when the association ratio favoured each response equally (ratio of $\frac{1}{1}$).
In reality, the fitted model predicted 72\% accuracy on such trials,
indicating that participants were also influenced
by structured taxonomic relationships.

Unlike Experiment 3, the current experiment revealed
that response times for correct responses were
marginally faster when the association ratio favoured the correct species
(i.e. when the associative and structured knowledge cued the same response).
There were, however, faster movement initiation times
for conflict trials in Experiment 3,
a finding that was not replicated here.
Of course, these two results are likely related:
initiation times are included in total response times,
and so faster initiation times under conflict in the previous experiment
may cancel out an overall effect on response time.
However, my analyses do not rest on these measures,
and so I will not attempt to interpret them further.

More interesting are the analyses of the cursor trajectories.
First, I found that initial cursor movements
were strongly predicted by associative knowledge,
and only marginally predicted by structured knowledge:
strength of association being equal,
participants are expected to move towards the correct species 62\% of the time.
After moving towards the correct species,
participants rarely
(20\% of the time when the species were equally associated)
changed direction to select the foil instead,
although they were more likely to do this
if the foil was more strongly associated than the correct species.
After initially moving towards the foil species, however,
participants were somewhat more likely to change direction:
they were predicted to do so 37\% of the time
when the association strengths were equal,
and more often when the correct species was more strongly associated.

From all of this, we can conclude three things.
First, associative knowledge predicts participants' movements
at every possible juncture, and unsurprisingly,
participants are more drawn towards a response option
when it is more strongly associated than the alternative.
Second, structured knowledge also plays a role at every juncture.
Participants were marginally more likely
to initially move towards the correct species
than would be expected based on associative knowledge alone,
and later, participants were in general more likely to change direction
to the correct species after initially moving towards the foil (37\%)
than to change to the foil species after
initially moving towards the correct one (20\%).
Third, once they started moving towards a response option,
participants in this experiment were unlikely to change direction.

Analysing the proportion of correct responses
where the cursor trajectory was classified as a reversal,
an unexpected trend emerged.
The data could be modelled using the association ratio as a predictor:
correct responses were more likely to involve
initial movements towards the foil
when the foil was more strongly associated than the correct species.
However, the data were slightly better fit by a model that
used the magnitude of this ratio
(how far it was from a ratio of $\frac{1}{1}$, in either direction),
resulting in the inverted U trend seen in Figure~\ref{fig:exp4_reversals}.

However, while this pattern was unexpected,
it can be understood in light of the transition probabilities, discussed above.
Participants were, first of all, more likely to
initially move towards the correct species
when the association ratio favoured this response.
Additionally, regardless of their initial movement,
participants were also more likely to ultimately select the correct species
when it was favoured by the association ratio.
Combined, these factors produce three kinds of trajectory.
When the association ratio strongly favoured the correct species,
participants moved straight towards that species, and selected it,
rarely moving to the foil at all,  and yielding few reversals.
When the ratio favoured the foil species,
participants who moved towards the foil usually ended up selecting it,
leaving few who moved to the foil before selecting the correct species.
When the ratio did not favour either species, however,
some participants  moved towards the foil initially,
and some of these changed their minds,
yielding a higher number of reversal trajectories.

Finally, the time course trends here are consistent with those found in Experiment 3.
As the association ratio becomes stronger in favour of the correct species,
participants became more likely to ultimately select this species,
but there were no striking changes in the trends leading up to these responses.
Additionally, consistent with the analysis of the transition probabilities,
even when associative and structured knowledge conflicted strongly,
there was no indication that participants were
first driven by associative knowledge,
and then by structured knowledge.
This is again consistent with the trends seen for Experiment 3.


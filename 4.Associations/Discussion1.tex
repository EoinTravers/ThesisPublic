
\subsection{Discussion}


 
In this experiment, participants had to generalise a biological property
from a base species to one of two other species:
one that belonged to the same taxonomic group as the base,
and one that did not.
On conflict trials, but not control trials,
there was a strong association between the base and foil,
so participants should have been drawn towards
selecting the foil species on these trials
if this irrelevant associative knowledge plays a role in inductive reasoning.
Replicating the core findings of \citet{Bright},
I found that participants were more likely to select the foil species
on conflict trials,
driven by associative knowledge.

Going beyond previous research, this experiment
allows us to draw inferences about how
these two kinds of knowledge interact.
\citet{Bright} also showed that
participants were more likely to select the foil response under conflict
if they had poor semantic inhibitory control,
or were placed under heavy cognitive load.
This suggests that drawing on the structured knowledge
needed to respond correctly on conflict trials
requires top-down executive functions.
It may be that the actual utilisation of structured knowledge
places demands on these executive processes, or alternatively,
that associative knowledge must be inhibited
before structured knowledge can be retrieved.
Perhaps consistent with the latter interpretation,
I found that on trials where participants did select the correct species,
they were more likely to initially move towards the foil species
on conflict trials, where the foil was strongly associated with the base.

Other findings, however, make this less clear.
The proportion of correct responses that were reversals, for instance,
was low compared to other experiments reported in this thesis:
8\% of control trials, and 16\% of conflict trials.
The current experiment also differed from Experiments 1 and 2
in that participants' response times were no slower
for conflict trials than for control trials.
Similarly, the participants initially moved towards
the foil species on only 35\% of conflict trials,
but after doing so, they only subsequently changed their minds
to select the correct species instead 34\% of the time.
Finally, the time course analysis showed that
even on conflict trials, participants never were more drawn
towards the foil species than the correct one.
Therefore, a more accurate description of these results is as follows.
In some cases participants' mouse movements
were initially driven by associative knowledge,
and later intervened upon on the basis of structured knowledge.
On the majority of conflict trials, however, participants either
moved directly to the species cued by structured knowledge,
or moved straight to the response cued by associative knowledge.

These results differ from those found in Experiments 1 and 2 in a number of ways.
However, before I attempt to interpret these results any further,
a limitation of this experiment must be noted.
Participants' performance on the post-test check,
which was used to ensure that they knew that
each correct response species belonged to the same biological group
as its corresponding base species, was extremely poor.
Specifically, participants' performance was not significantly above chance
when it came to knowing that the following species belonged to the same group:
dolphins and llamas,
monkeys and seals,
snails  and octopuses,
bananas and tulips,
penguins and chickens,
mice and goats,
and Orca whales and cows.
As a result, data from these 7 of the 14 stimuli sets
were excluded from the analysis.
Given that the purpose of this chapter is to study
conflict between associative and structured knowledge,
such poor taxonomic knowledge may be problematic.
Consider, for instance, 
a participant who is drawn towards 
generalising a gene from dolphins to cod,
rather than from dolphins to llamas.
If this participant knows that dolphins and llamas are mammals,
and cod are not, then we can infer that this attraction
is driven by unstructured associative knowledge.
However, if this participant believes that
dolphins and cod belong to the same taxonomic group,
then both their associative and structured knowledge
would support the same inference.
In this case, it is difficult to know what form of knowledge
they are drawing upon.

Therefore, In Experiment 4, I attempted to replicate these findings conceptually,
using new stimuli for which the appropriate taxonomic relationships are more obvious.
Furthermore, Experiment 3 relied on association ratings
collected by \citet[][Chapter 2]{Crisp-Bright2010}
from a different pool of university students,
and only allowed for the manipulation of the foil species,
which was either strongly associated with the base,
according to the prior ratings,
or (assumed to be) not associated.
In Experiment 4, in contrast,
I collected association ratings for each species pair
from each participant, after they had completed the rest of the experiment.
In this way, it was possible both to
investigate the relationship between association ratings
and participants' choices and cursor movements in a more nuanced way,
and to use each participants' actual beliefs about
the strength of association between species,
rather than aggregate ratings from a separate pool of participants,
in the analyses.


\section{Introduction}

Chapters 3 and 4 explored conflict between
different kinds of information in category-based induction.
In the reasoning literature, however, there is another form of conflict
that has received considerably more attention,
between two qualitatively different types of cognitive process.
Such conflicts are the domain of dual process theories of reasoning
\citep{Evans2013a,Evans2006,Sloman1996,Kahneman2011},
which typically distinguish between
fast, effortless, and autonomous \emph{Type 1 processes},
and slower, demanding, and controlled \emph{Type 2 processes}.

These dual process theories have been discussed in detail in Chapter 1.
However, it is worth noting again that it is only in recent years
that much attention has been given to questions of
how Type 1 and Type 2 processes interact during reasoning.
Two questions in particular are relevant to the current chapter.
First, it is not yet clear how conflict between these processes is resolved,
with \citet{Evans2007a} outlining three possibilities:
a \emph{pre-emptive conflict resolution account}
\citep[e.g.][]{Klaczynski2000,Klaczynski2004,Chaiken1987},
by which one or other process is selectively activated for a given decision,
a \emph{default-interventionist} account
\citep[e.g.][]{Evans2006,Kahneman2002}
where Type 1 processes provide default responses,
which are inspected, and sometimes overridden by Type 2 processes,
and a \emph{parallel-competitive} model
\citep[e.g.][]{Sloman1996,Sloman2014},
where both kinds of process are activated simultaneously,
and compete to dictate the response given.

A second question, spurred by the intuitive logic theory
\citep{DeNeys2012,DeNeys2014a}),
concerns the usual assumption in dual process theories that,
while Type 1 processes implement the heuristics that
lead to systematic biases in human reasoning \citep{Kahneman2005,Kahneman2011},
logical reasoning requires the engagement of Type 2 processes.
According to this account, Type 1 processes can
simultaneously generate both logical and heuristic responses,
and as a result that even when we produce heuristic responses,
we implicitly and automatically detect the conflict
between these multiple Type 1 responses.
As discussed in Chapter 1, a number of seemingly implicit measures
have supported this \emph{implicit conflict monitoring} prediction:
on tasks in which the heuristic and logically-valid response conflict,
even when producing heuristic responses, participants are, for instance,
slower to respond \citep{DeNeys2008},
less confident in their heuristic responses \citep{DeNeys2013a},
sweat more during reasoning \citep{DeNeys2010},
look more at the logically-relevant information \citep{DeNeys2008},
and even appear to ``like'' logically valid syllogisms more than invalid ones
\citep{Morsanyi2012}.
Explicit measures such as participants' verbal reports,
on the other hand, rarely show such evidence of conflict \citep{DeNeys2008},
suggesting that conflict is detected by Type 1,
rather than Type 2 processes.
These questions are relevant to reasoning in every domain.
However, the \emph{base rate neglect} paradigm in particular
is extensively used as a testing ground
both for dual process accounts more generally,
and for the intuitive logic theory specifically.
It is this paradigm I use here.


\subsection{Base rates in Reasoning}

In the psychology of judgement and reasoning,
few phenomena have received as much attention as base rate neglect.
The phenomena was introduced by \citegap{Kahneman1973}{'s} \emph{Tom W.} problem.
In its original form, participants read a description of Tom,
a randomly chosen student:

\begin{quote}
  Tom W is of high intelligence, although lacking in true creativity. He
  has a need for order and clarity, and for neat and tidy systems in
  which every detail finds its appropriate place. His writing is rather
  dull and mechanical, occasionally enlivened by somewhat corny puns and
  by flashes of imagination of the sci-fi type. He has a strong drive
  for competence.  He seems to have little feel and little sympathy for
  other people and does not enjoy interacting with others.
  Self-centered, he nonetheless has a deep moral sense.
\end{quote}

Participants were given a list of undergraduate degrees,
and variously required to indicate what proportion of American students
were enrolled in each degree (the \emph{base rate} of each degree),
how similar Tom W. was to a typical student of each subject,
or \emph{how likely Tom W. was to be studying for each degree}.
\citet{Kahneman1973} found that likelihood judgements for each degree
correlated almost perfectly (r = .97) with similarity judgements
(Tom sounds like a typical computer science student)
but were unrelated to the base rate judgements
(more students study the humanities than anything else).
In other words, when asked to make a decision about likelihood,
participants responded with a judgement of similarity, or \emph{representativeness}
and completely ignored their knowledge about the
statistical base rate for each degree.
Thus, \citegap{Kahneman1973}{'s} participants, like many since,
displayed \emph{base rate neglect}.

Following this first demonstration, researchers have created
a range of base rate problems, where statistical base rate information is available,
but typically ignored or underweighted in favour of other evidence.
One class of problem pits base rate probabilities
against statistics directly about the object at hand
This chapter, however, focuses on a different kind of base rate problem,
where base rates are typically ignored in favour of qualitative information,
such as a description that allows participants to
base their response on how representative something or someone is
of a given category \citep{Kahneman2002,Tversky1974,Kahneman1973}.
In their original forms, such problems were typically quite complex,
such as the example seen at the start of this chapter,
from \citet{Kahneman1973}.
In the same paper, they present a simpler test of
participants' use of base rates.
Participants were told of a group containing
70 engineers, and 30 lawyers (or 30 engineers, and 70 lawyers),
and presented with a description of a randomly chosen
member of this group, for instance:

\begin{quote}
  Jack is a 45-year-old man. He is married and has four children. He
  is generally conservative, careful, and ambitious. He shows no
  interest in political and social issues and spends most of his free
  time on his many hobbies which include home carpentry, sailing, and
  mathematical puzzles.
\end{quote}

Participants were asked to indicate the probability that
Jack was either an engineer, or a lawyer.
Across five descriptions, participants' ratings were
minimally influenced by the base rates.
When the sample consisted of 30 engineers and 70 lawyers,
participants on average said that there was a 50\% probability
of Jack being an engineer.
When it consisted of 70 engineers and 30 lawyers,
the average probability was 55\%.
Therefore, the base rate information was almost completely ignored.
Even with an uninformative description (but not with a person
for whom no description at all was provided),
base rate information was used minimally, or not at all.

More recently, base rate neglect,
along with many other phenomena in the heuristics and biases literature,
has been reinterpreted in terms of dual process accounts of cognition
\citep[see][]{Kahneman2005,Kahneman2002,Kahneman2011,Barbey2007}.
From this perspective, Type 1 processes are responsible for responses
based entirely on representativeness,
while Type 2 processes are required to override the description-cued response,
and to integrate this information with base rates/prior probabilities.
Of course, as noted by \citet{Stanovich2000},
use of base rate information requires that participants
a) are aware of its relevance \citep[see also][]{Bar-Hillel1980},
b) are motivated to attempt to use it,
c) have the cognitive capacity to hold
both kinds of information in working memory, and
d) can inhibit the intuitively appealing non-base rate response.

More recently still, \citet{DeNeys2008} introduced
a simplified version of the lawyer-engineer problem
that, like the original, placed base-rates
and representative descriptions directly in conflict.
Participants were told about a different sample on each trial,
that contained, for instance, 955 of one social group and 5 of the other.
They were again given a description of a ``randomly chosen'' person from that sample.
Their task, in this simplified forced-choice paradigm,
was to indicate which group they thought the person described belonged to,
rather than give a probability estimate,
as in the original version of the task.
For each trial, the base rate and description either
suggested the same response, or conflicted.
While protocol analysis of participants' ``thinking aloud''
during reasoning revealed little evidence that they
explicitly considered the base rate information,
they were nevertheless more likely to correctly remember
base rate information from conflict trials,
where base rates and descriptions supported different responses,
suggesting greater processing of such information those trials,
even if not to a degree that participants were explicitly aware of.
In a second experiment, \citet{DeNeys2008} also showed that
even when consistently giving the response consistent with the description,
participants took longer to do so, and were more likely 
to look back at the base rate information, when the cues conflicted.
Again, this indicates that participants processed the base rate information,
and experienced conflict as a result,
on these trials, even when it rarely affected their responses.

The simplified base rate task presented by \citet{DeNeys2008}
has become a popular tool for testing theories of conflict in reasoning.
In an fMRI study, \citet{DeNeys2008a} presented participants with 
conflict and no-conflict problems,
along with additional control conditions in which either
the description was uninformative,
or the base rates were equal (500:500).
Behaviourally, they found that when the cues conflicted,
participants selected the base rate-cued response on 45\% of trials.
Additionally, they found that both kinds of response
were slower under conflict than any other condition,
suggesting that even when participants based their decisions 
under conflict on the description provided, rather than the base rate,
they were to some degree slowed by the conflict between the two cues.
\citet{DeNeys2008a} also presented neuroimaging results supporting this interpretation,
with greater activation of the anterior cingulate cortex,
thought to reflect the detection of conflict \citep{Botvinick2004a},
on conflict trials regardless of the response given than in any other condition.
They also found greater activation of the dorsolateral prefrontal cortex,
usually an indicator of inhibitory processes \citep{Aron2004},
for base-rate consistent responses under conflict
compared to description consistent responses.
This suggests that responding based on the base rate
requires inhibition of the description-cued response.

A number of other studies have used
this forced-choice version of the base rate paradigm.
\citet{Franssens2009} showed that while placing participants
under secondary load led them to be
even less likely to select the base rate-cued response,
their recognition memory for the base rate information was unaffected,
suggesting that this conflict detection is not cognitively demanding.
In a similar vein, \citet{DeNeys2009a} asked participants to complete 
a lexical decision task after conflict and control base rate trials.
They showed that in the lexical decision task,
participants were slower to identify words used
in descriptions that conflicted with base rates.
This effect was found even for those participants
who predominantly failed to give the base rate-cued response on such trials,
suggesting that even these participants were to some degree
attempting to inhibit the description, although unsuccessfully.
\citet{DeNeys2011b} also showed that participants were
less confident in their description-cued responses
when the base rate disagreed with them,
again suggesting some awareness of this conflict.

Collectively, these results provide support for 
De Neys' \citep[i.e.][]{DeNeys2008,DeNeys2008a} claim
that heuristic reasoners are aware that their responses are biased,
at least on the base rate neglect paradigm
(see Chapters 1 and 6 for evidence for this claim in other paradigms).
In the last few years, a number of studies
have provided evidence for a stronger claim made by \citet{DeNeys2012,DeNeys2014a},
that intuitive Type 1 processes can simultaneously cue both
heuristic (description-based, in the base rate paradigm)
and logical (base rate-based) responses.
Of course, this claim goes against classical dual processes accounts
of base rate neglect \citep{Barbey2007,Kahneman2005},
which claim that Type 2 processes underlie base rate-based responding.

In one study,
\citet{Pennycook2012b} presented participants with
problems featuring no base rate information,
and problems with base rates that were either consistent or conflicted with the descriptions.
Participants were  asked to judge the probability
of the person described belonging to one or other group.
They then analysed the distribution of these probability judgements.
Using problems without base rate information as a baseline,
they found that when base rates agreed with the descriptions,
the probabilities were shifted in the direction
consistent with both of these cues.
When the base rates disagreed with the descriptions, however,
the distribution of the probabilities became bimodal,
as participants gave either a rating consistent with the description,
or one consistent with the base rate.
Crucially, they argue that this pattern of results
requires that participants always process the base rate,
in order that they should integrate it with the description when they agree,
or decide to rely on one or other cue when they conflict.

A particularly compelling result is reported by \cite{Pennycook2014},
who presented participants with base rate problems
where the cues either conflicted or agreed,
and instructed them to either base their response
on the base rate, or on the descriptions.
They analysed the probability judgements participants gave,
their confidence in these judgements, and their response times.
They found that conflict occurred in both directions.
Regardless of what cue they were told to use,
participants probability judgements were
affected by manipulations of the other cue.
These interference effects persisted
even when participants responded under time pressure,
where they were also slower to respond under conflict,
regardless of what cue they were using.
Recall that classical dual process accounts \citep{Barbey2007,Evans2006}
hold that processing of descriptions
is quick, effortless, and obligatory, driven by Type 1 processes,
while processing of base rates
is slow, effortful, and deliberate, driven by Type 2 processes.
These results would suggest that processing both kinds of information
can be quick, effortless, and obligatory.
\citet{Handley2015} further develop this idea,
arguing that much of the information processing
usually ascribed to Type 1 or Type 2 processes
can actually be performed by both sets of processes.


\subsection{What Mouse Tracking can add to the Debate}

The work cited above goes some way towards revealing the processes
that underlie base rate neglect and respect.
However, questions remain, which in this chapter
I will attempt to address using the mouse tracking paradigm.

To date, work in the base rate problem has been typically based
on the analysis of certain form of data,
specifically, binary choices, probability judgements,
confidence ratings, and response latencies,
as well more subtle measures such as
recall of base rate information \citep{DeNeys2008},
or subsequent lexical decision times for words from the description \citep{DeNeys2009a}.
The mouse tracking paradigm, however, differs from these methods
in that it both allows for the detection of conflict 
between competing responses on a single trial,
and provides some insight as to the nature of this conflict ---
for instance, initial movements towards one response option that are later overridden,
or changes in the time taken to select a response,
in the absence of such cursor reversals.



Naturally, different accounts of the base rate paradigm
yield different predictions about what mouse tracking will reveal here.
Usually, studies of base rate neglect
are couched in terms of dual process theories.
However, as discussed in Chapter 1, such theories come in many varieties.
First, selective, pre-emptive conflict resolution accounts
\citep[e.g.][]{Klaczynski2000}
would predict that participants engage either Type 1 processes
sensitive to the contents of the description,
\emph{or} Type 2 processes also sensitive to the base rate information.
Such an account would predict that participants take longer
to give the base rate-cued response (using Type 2 processes)
than to give to the description-cued response (using only Type 1 processes).
However, because only one or other kind of process is activated,
it would not predict that participants should be conflicted.
Therefore participants giving the base rate-cued response
should not be slower to do so,
or more likely to move towards the alternative option,
if the description cues the alternative response.
Likewise, participants giving the description-cued response
should not be influenced by manipulations of the base rate.

A default-interventionist account
\citep{Evans2007,Evans2006,Kahneman2002,Kahneman2005}, on the other hand,
would predict that participants are initially driven by Type 1 processes
to give description-cued responses.
However, this default may be overridden by Type 2 processes
that take into account base rate information later in the process.
The predictions made by such an account are clear:
in addition to the latencies discussed above,
early cursor movements should be driven by the contents of the description,
but the base rate may become influential later.
Therefore, participants giving the base rate-cued response
should be slower to respond or initially drawn towards the alternative option
when the description cues it instead,
and faster when the description agrees with the base rate.
However, because these accounts hold that
Type 2 processes are only engaged when they override Type 1 responses,
participants responding based on the description
should be unaffected by manipulations of the base rate.

A number of previous findings, however, discussed above,
do suggest that conflict occurs in both directions ---
including many results indicating that participants
are sensitive to a conflict with the base rate
even when giving the description-cued response
\citep{DeNeys2008,DeNeys2008a,Pennycook2012a},
and \citegap{Pennycook2014}{'s} finding
that regardless of which cue participants were told to use,
their judgements were influenced by the other cue.
One theory that may account for such findings
is the parallel-competitive dual processes theory \citep[i.e.][]{Sloman1996,Sloman2014},
which claims that both Type 1 and Type 2 processes
are activated simultaneously, and compete to produce responses.
Alternatively, and more commonly, these findings
can be seen as evidence for the intuitive logic theory \citep{DeNeys2012},
by which account Type 1 processes can simultaneously cue
both description-based and base rate-based responses,
leading to both the implicit detection of conflict,
and the interference when the cues conflict,
regardless of the response actually given.
The account proposed by \citet{Handley2015} is in many ways similar.
They suggest that both base rates and descriptions
can be processed by both Type 1 and Type 2 processes.
All of these accounts yield largely the same predictions here:
under conflict, participants should be drawn to the competing response,
regardless of what response they ultimately give.
In other words, beyond the effect of
descriptions on base rate-cued responses predicted by the other accounts,
they predict that participants giving the description-cued response
should be affected by manipulations of the base rate,
and show greater evidence of conflict when it disagrees with the description.
These accounts do, however, allow for some asymmetries:
the description-cued response is thought to be \emph{stronger},
either because it is generated by the faster Type 1 processes \citep{Kahneman2005,Kahneman2002},
or because it is the \emph{prepotent} of the two Type 1 generated responses,
and so its interference on base rate-cued responses
may be greater than the interference in the other direction \citep[see][]{DeNeys2012}.
In this experiment, I use the mouse tracking paradigm to test these predictions.




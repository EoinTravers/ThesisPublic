
\section{Results}

\subsection{By-trial analyses}

After excluding data from 3 participants 
who did not complete the experiment within the 15 minutes allocated, 
and 7 trials with response times greater than 100 seconds (.6\% of the total), 
participants selected the correct option on 79.5\% of non-conflict problems.
On the conflict problems, the correct option was chosen 36\% of the time,
the heuristic option 58\%, and one of the foils 6\% of the time.

In the first stage of the analysis, 
I calculate a number of summary statistics for each trial, 
and compare these between problem types, and between responses. 
The measures were response time, 
the distance travelled by the mouse cursor
(scaled so that a straight line from the start point
 to the response corresponds to 1 unit),
the number of times the cursor was moved during a trial
(with movements defined as windows of 100 msec or more in motion, 
separated by 100 msec or more not moving), 
the closest proximity achieved between 
the cursor and the non-chosen option
(closest proximity to the heuristic response option 
on trials where the correct option was chosen, and vice versa). 
These measures were compared using linear mixed models, 
with crossed random intercepts for each participant, 
and each problem \citep{Baayen2008}.
Response latencies, and the distance travelled by the mouse cursor
were log-transformed to normalise their distributions, 
and a generalised mixed model with a Poisson link was used
to model the number of movements. 

Consistent with a dual process interpretation, 
whereby heuristic responses are generated by Type 1 processes,
and correct responses under conflict by Type 2 processes,
for conflict problems there was greater evidence of conflict
across all measures when participants gave the correct response (N = 181) 
than the heuristic one (N = 297).
The average time to respond was 27.3 seconds (SD = 16.3) for correct responses, 
and 21.0 seconds (SD = 13.4) for heuristic responses
($e^{\beta}$ = 114\%, CI = [102\%, 129\%],
t(470.8) = 2.349, p = .0192).
The mouse cursor travelled a greater distance 
before selecting a correct option (6.11 times the minimum needed distance, SD = 5.6) 
than an heuristic option (5.66 times, SD = 4.74;
$e^{\beta}$ = 116\%, CI = [102\%, 133\%],
t(298.4) = 2.267, p = .0241). 
There were also more cursor movements on
trials in which the correct response was given (5.4, SD = 4.8) 
than when the heuristic response was given (4.9, SD = 4.5; 
$e^{\beta}$ = 1.15, CI = [1.02, 1.29],
z = 2.337, p = .0195).
Finally, the minimum distance between 
the cursor and the heuristic option on trials in which the correct option was chosen 
was on average 49\% of the display width (SD = 24\%),
significantly less than the minimum distance between
the cursor and the correct option 
on trials in which the intuitive option was chosen (55.5\%, SD = 18\%,
$e^{\beta}$ = 0.92, CI = [0.89, 0.96],
t(72.1) = 4.119, p < .0001).

Most tests of the intuitive logic model compare
correct responses on no-conflict problems with
heuristic responses on conflict problems, 
on the basis that heuristic, Type 1 processes should cue both kinds of response, 
but the chosen response conflicts with normative principles on conflict problems only. 
Evidence for the intuitive logic model 
therefore comes from results which indicate 
greater conflict for heuristic responses to conflict problems (N = 404). 
However, there was no such reduction in conflict for no-conflict problems
on any of the applicable measures:
response time (23.1 seconds, SD = 15.3; t(14.3) = 0.222, p > .8),
distance travelled (5.6, SD = 5.0; t(15.0) = 0.359, p > .7) 
and number of movements per trial (5.2, SD = 4.6; z = 0.064, p > .95).

Following previous intuitive logic studies
\citep[e.g.][see also \citealp{Pennycook2015}]{DeNeys2011b,Mevel2014},
I also calculated the number of heuristic responses given
by each participant on conflict problems, 
and categorised each participant as either
``majority heuristic'' (3 or 4 heuristic responses out of four, 53 participants)
or ``minority heuristic'' (0 to 2 heuristic responses, 75 participants). 
I entered this measure as a participant-level predictor in the models, 
but found that it was not involved with any interactions in the analyses above
(t's < .9, p's > .4). 
I also repeated this analyses for the most (4 heuristic responses) 
and least (one heuristic response) biased reasoners only, 
again finding so significant interactions (t's < 1.1, p's > .25).
Therefore, these analyses revealed no evidence for logical intuitions
in either biased or unbiased participants.


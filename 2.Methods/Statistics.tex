
\section{Statistical Analyses}\label{sec:statistics}

Finally, I will use the remainder of this chapter
to outline some of the statistical methods used in this thesis.
Again, by covering this material here,
I hope to avoid repeating it throughout the later chapters.

\subsection{Mixed Effects Models}\label{subsec:statistics-lme}

Throughout this thesis, I use \emph{mixed effects} regression models
\citep{Gelman2007, Baayen2008}
--- otherwise known as linear or logistic mixed models (LMMs), hierarchical regression,
multilevel regression, and random effects regression ---
to analyse my data.
I will not provide a full tutorial on these methods here
\citep[see instead][]{Gelman2007, Baayen2008,Quene2008,Bates2015a}.
However, in this section I will explain my motivation in using these approaches,
and provide sufficient introduction so that readers
can follow the results presented later.
Unless otherwise stated, all analyses were conducted
using the R programming language \citep{RCoreTeam2015},
and mixed effects models were fit using the lme4 package \citep{Bates2015a}.

A typical psychology experiment records
several data points from each participant, one per trial.
For illustration, I will use the case of an experiment where
participants $p_1, p_2, ..., p_P$  perform a task in two conditions, $c_1$ and $c_2$,
and their response times $rt$ are recorded on each trial.
In a classical analyses, we would calculate the mean $rt$
for each participant in each condition,
and compare each participants' means from $c_1$
to those from $c_2$ using Student's t test.
Furthermore, there will be individual differences in $rt$ in general,
as some participants generally respond faster than others ---
this data is clustered within participants, in other words.
Therefore, we can improve our statistical power
by using a paired t test,
which accounts for these individual differences in baseline $rt$.

However, aggregating data in this way is not always appropriate
\citep[see][]{Baayen2008}.
Firstly, it discards a great deal of information,
and as a result the statistical power of our tests can be diminished.
Secondly, in averaging over multiple stimuli,
we ignore any systematic clustering by stimuli (or items) in our data:
some stimuli may produce faster responses than others in general, for example.
In linguistics, where items usually correspond to specific words or phrases,
this issue is traditionally addressed by running two analyses,
one aggregating by participant, and one by item,
and reporting the results of both \citep{Clark1973}.

A more serious problem with aggregation
is that it can involve unbalanced data.
In many analyses, we only include trials which meet certain criteria,
for instance only analysing response times for correct responses.
In reasoning research, where participants can often produce many different inferences,
this is particularly common.
If one participant gives few correct responses,
their average response time will be estimated from only a few trials,
and so will be a noisy measure.
In the classical analysis, however,
this datum is given as much weighting
as an average response time from a participant
who responded correctly on every trial.
When, as often is the case, there is some relationship between,
for instance, accuracy and response times,
this can cause systematic biases in our analysis,
hiding true effects and potentially creating spurious ones \citep{Baayen2008}.

Aggregation is also inappropriate
when the dependent variable is not a continuous quantity such as $rt$.
Most notably, when analysing proportions and percentages (e.g. accuracy),
aggregate data systematically violate the assumptions of t tests/ANOVA,
in a way which cannot be remedied by transforming the data \citep{Warton2010}.
These issues with aggregation can be avoided, however,
by fitting regression models to the data from each trial, rather than the aggregates.

As every undergraduate psychology student is taught,
the standard (unpaired) t test,
like the between-subjects ANOVA and many other classical tests,
is a special case of linear regression.
The regression equivalent of the unpaired t test for this experiment is given as

\begin{equation}
  \label{eq:lm1}
  rt_i = \alpha + \beta * Condition_i + \epsilon
\end{equation}

$rt_i$ is the $rt$ on trial $i$, and is the sum of
$\alpha$ (the mean $rt$ in baseline condition $c_1$),
$Condition_i$ (a dummy variable representing the condition on trial $i$,
set to $0$ to represent $c_1$ and $1$ to represent $c_2$),
$\beta$ (the difference in $rt$ between the conditions),
and the normally distributed error term $\epsilon$.

However, like the unpaired t test,
this model does not account for individual differences in $rt$.
To do this, we can extend Equation~\ref{eq:lm1}
so that the intercept $\alpha$ is allowed to differ for each participant.
This is, in effect, a within-subjects regression model.
The result is a \emph{mixed effects model},
in that it includes both \emph{fixed effects} ---
the overall intercept term $\alpha$,
and the overall effect of condition $\beta$ ---
and \emph{random effects} ---
the individual differences around $\alpha$.
Such a model is said to have \emph{random intercepts for each participant}.

Mixed models are even more flexible than this, however.
We may also want to allow for differences between each item in $rt$.
This can be done by including \emph{crossed random intercepts} \citep{Baayen2008}
for both participants and items,
allowing $\alpha$ to vary for each participant, and for each item,
while still calculating the overall intercept term.
These random effects are not restricted to the intercept either.
The $\beta$ term reflects the difference between the two conditions.
It may be that this difference is greater for some participants than others,
or greater for some items than others, if the same items are used in both conditions.
Therefore, we can also add random effects of condition for each participant,
for each condition, or for both, as desired.

\subsection{What Random Effects?}\label{subsec:statistics-random}

At this point, it should be noted that there is still active debate
as to what random effects should be included in a model.
\citet{Barr2013} argue for the use of \emph{maximal} random effects structures,
where random effects are included wherever the design of the experiment allows.
However, even with simple designs such a model cannot always be fit in practice,
as the amount of data required increases exponentially with the number of parameters to be fit.
Instead, \citet{Bates2015} advise using \emph{parsimonious} models ---
fitting the maximal model, and then removing the parameters
that account for least variation as necessary until the model can be fit.
It is this approach I have taken in this thesis,
and in each results section I specify the random effects structure used for the analysis.


\subsection{Data Transformations and Binary Outcomes}\label{subsec:statistics-tranform}

Being regression models, mixed effects models
can also make full use of the data transformations
and link functions used in regression modelling.
In this thesis, I use two such tools.
First, it is expected that latency data,
such as response times, movement initiation times,
and in Chapter 5, reading times,
will have a log-normal distribution:
a distribution like a normal distribution
that has been log-transformed.
Therefore, to analyse these measures I simply log-transform them,
and apply a standard linear mixed effects model.
Second, several of the measures I analyse,
including reasoning accuracy,
and the number of reversal trajectories per condition, are binary proportions.
These can be analysed by using logistic, rather than linear, mixed effects models.

\subsection{Categorical Predictors}\label{subsec:statistics-catgorical}


Just as the ANOVA is a special case of linear regression,
mixed effects models can be used
to evaluate the main effect of a categorical predictor
in these complex repeated-measures designs.
In the classical ANOVA, we use the Sum of Squares%
\footnote{More accurately, the sum of the squared deviations from the model fit.}
of models fitted with and without that predictor to  compute an F statistic,
which reflects the main effect of that predictor.
However, with mixed effects the degrees of freedom (DFs) for this F statistic
are not easily defined, and so it is difficult to calculate the appropriate p value.
In mixed effects models, we therefore compare
the deviance of models ($-2$ x their log-likelihoods)
fitted with and without the predictor,
and use a chi-squared test,
with DF corresponding to the number of terms added by larger model
(adding a categorical predictor with three levels adds 2 DF)
to find a p value for that predictor's main effect.

\subsection{Reporting}\label{subsec:statistics-reporting}

Of course, the output of mixed effects regression models
differs from that of a simple t test.
In most analyses, I will be investigating the effect of one or more
binary variables, such as condition, on the dependent variable.
This is quantified by the regression slope ($\beta$) for that variable,
and so for each comparison I will typically report
the means and standard deviations in each condition,
$\beta$ and its 95\% confidence intervals,
reflecting the estimated difference between the conditions,
and finally a t test comparing $\beta$ to 0.%
\footnote{
  In linear regression, the DF for this test
  would simply be the number of data points minus the number of variables.
  When random effects are included, it is not possible to define the DF in this way.
  In this thesis, I  take the conservative approach
  of using Satterthwaite’s approximation \citep{Satterthwaite1946} for the DF,
  as implemented in the lmerTest \citep{Kuznetsova2015} package for R.
  For logistic mixed models, the t test is replaced by a z test,
  and so it is not necessary to calculate DF.
}

For analyses where the dependent variable was log-transformed (e.g. response times),
the exponentiated regression weight $e^{\beta}$ is reported instead.
This indicates the percentage change in the dependent variable
between the conditions.
For instance, if the mean response time in condition $c_1$ is 1,000 msec,
and the effect of condition is $e^{\beta} = 110\%$,
response times in condition $c_2$ are
10\% greater than those of condition $c_1$, for an average of 1,100 msec.
$e^{\beta}$ is also reported for the logistic analyses,
where it reflects the multiplicative change in the odds of the outcome in question.
For example, if participants
gave the wrong response on 20\% of trials in condition $c_1$,
the odds ratio of wrong to right responses in this condition is $\frac{20\%}{80\%} = 0.25$.
An effect of $e^{\beta} = 1.5$ would mean
the odds were one and a half times greater of doing so in condition $c_2$;
odds of $0.25 * 1.5 = 0.375$,
which correspond to a probability of $\frac{0.375}{1 + 0.373} = 27.3\%$.

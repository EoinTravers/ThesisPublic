\section{Thesis Overview}

This thesis is about conflict in reasoning.
Many previous investigations of such conflict
have used methodologies that focus on
the end product of the reasoning processes,
rather than the unfolding process itself.
In this thesis, I apply a method new to the reasoning literature,
the mouse tracking paradigm, which allows not only analysis of participants' responses,
but also their movements on the way to giving these responses.
In doing so, the experiments reported herein reveal much about
when conflict occurs during reasoning,
as well as helping to answer questions about the qualitative nature of this conflict.

The objectives of this thesis are threefold.
First, by applying the mouse tracking paradigm to a number of reasoning tasks,
I will show directly when conflict occurs during reasoning.
Second, by analysing the time course of participants' mouse cursor trajectories,
it is possible to infer at what point in time
participants are driven by various competing influences in reasoning.
Finally, the paths followed by these mouse cursors
can reveal not only the presence of conflict,
but something of the qualitative nature of this conflict
and the interaction between the various processes.

\subsection{Thesis Structure}

The remainder of thesis is organised as follows.
In Chapter 2, I discuss the technical details of
the mouse tracking paradigm used in subsequent experiments,
and the analyses used to explore the data it generates.

Following this, I report the results of six experiments, across four chapters.
Chapters 3 and 4 concern conflict between different
sources of information during inductive reasoning.
In Chapter 3, I present versions of the inductive triad task
that pit information present in perceptual
cues
against more abstract conceptual knowledge.
This is done both for real categories in the natural world (Experiment 1)
and for artificial categories that participants learned in the lab (Experiment 2).

Chapter 4, in a similar way, pits associative knowledge
against structured, relational knowledge,
adapting the triad task presented by \citet{Bright}.
Experiment 3 directly adapts
\citegap{Bright}{'s} stimuli for the mouse tracking paradigm,
while Experiment 4 replicates Experiment 3 using different stimuli,
and a design that allows for a more fine-grained analysis
of how the two forms of knowledge interact.

Chapters 5 and 6 deal with conflict in reasoning from a dual process perspective,
and pit fast, automatic Type 1 processes
against slow, effortful Type 2 processes.
In Chapter 5 (Experiment 5), I present a mouse tracking version of
one of the most popular tasks in the dual process/heuristics and biases literature:
the base rate neglect paradigm,
where participants make judgements that can be influenced by
stereotypes about social groups, thought to be mediated by Type 1 processes,
and statistical information, traditionally thought to
require Type 2 processes to process.


Chapter 6 (Experiment 6) departs somewhat from the previous work,
as I present a four-option multiple choice version
of the Cognitive Reflection Test \citep{Frederick2005},
a paradigmatic dual process/heuristics and biases task
that pits incorrect, intuitively appealing, heuristic Type 1 responses
against the less obvious correct response, reached using Type 2 processes.
Unlike the previous experiments, this involves analysis of
participants' mouse movements over up to 60 seconds,
as the meandering of the cursor during complex reasoning
reveals participants tentative beliefs about each response.

This thesis ends, in Chapter 7, with a general discussion,
in which I talk about what has been learned over
the course of the previous six experiments,
and discuss implications for
theories of reasoning,
accounts of conflict in cognition,
and users of the mouse tracking paradigm.


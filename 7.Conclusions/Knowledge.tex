
\section{Knowledge and Reasoning}\label{sec:label}

%% Leaving aside this broader framework for a moment,
%% my results have particular implications
%% for a number of theories of inductive reasoning in particular.
In Experiments 1 and 2 (Chapter 3) and 3 and 4 (Chapter 4),
I looked at what information people drew on during inductive reasoning,
using triad tasks.
In these, participants were asked to project a property
from a base category, known to have the property in question,
to one of the two other categories.
Above, I argued that this task requires that participants represent
the three categories in working memory,
and then draw on information about each category
to decide to which of the two response categories the property likely generalises.
Thus, the framework above provides
a mechanistic account of the operations proposed
by \citegap{Bright2014a}{'s} hybrid theory of induction.

In Experiments 1 and 2,
I pitted perceptual cues against conceptual knowledge,
and in Experiments 3 and 4,
I pitted associative knowledge against structured knowledge.
An interesting question that arises at this point, therefore,
is if these two kinds of conflict really unfold in the same way.
In Experiments 1 and 2, the interaction of
perceptual cues and conceptual knowledge was straightforward:
perceptual cues drove participants' motor output early in reasoning,
and conceptual knowledge was retrieved later,
causing participants driven towards the foil response
to sometimes change direction to select the right one instead.
In Experiments 3 and 4 the picture was less clear.
On the one hand, participants did sometimes initially move towards the foil option
before giving the correct response,
and a greater proportion of correct responses were reversals under conflict.
On the other hand, these reversals were considerably rarer
than in Experiments 1 and 2,
as participants for the most part moved towards one other response option,
and then selected it.

These differences can be captured by two of the parameters
in my analysis of the transition probabilities of cursor trajectories:
$1 - \alpha$, the probability of initially moving towards the foil option,
which was higher for perceptual cues (\tildetext50\%)
than for associative knowledge (\tildetext35\%),
and $\gamma$, the probability of overriding an initial movement towards the foil,
which was again higher for perceptual cues (\tildetext50--80\%)
than associative knowledge (\tildetext25--35\%).

One possibility here is that
the differences between the two sets of experiments are only quantitative.
It may be that participants are very prone to
draw on perceptual cues during reasoning,
but that they are also relatively good at
overriding the influence of these cues,
doing so, at minimum, on 50\% of trials.
Conversely, it may be that
associative knowledge may only be retrieved some of the time,
but participants who do retrieve this knowledge are
considerably less likely to subsequently inhibit or override it.
From this point of view, the phenomena studied in
both Experiments 1 and 2 and in Experiments 3 and 4
are examples of conflict occurring as different Type 1 processes
attempt to project information to working memory.

An alternative possibility, however, is that
the difference between the two sets of experiments is qualitative.
An extensive body of research \citep[see][for a review]{Goodale2004}
shows that information passes through the human visual system in two streams.
A dorsal stream leads from the visual cortices
to the prefrontal cortex and allows us to
consciously perceive stimuli and to hold them in working memory.
A ventral stream, however, leads directly to the motor cortex
and allows our actions to be informed by perceptual cues
without conscious awareness.%
\footnote{
  In the extreme, damage to ventral system
  can leave individuals able to respond to visual stimuli,
  without being consciously aware of seeing them,
  a condition known as \emph{blindsight} \citep{Weiskrantz1986,Salti2015}
}
It could be the case that
rather than participants seeing the visual cues in Experiments 1 and 2,
representing this information in working memory,
and reasoning from there,
this information may have been passed directly
from the perceptual system to the motor cortex.
In this case, these results could also be interpreted as a dual process phenomenon,
if we allow that this perceptual-action coupling constitutes a Type 1 process.

On the basis of the current data,
it is not possible to differentiate between these two possibilities.
A possible future approach, however,
may be to investigate the relationship between
performance on both of these versions of the triad task
and measures of different kinds of inhibitory control.
According to \citet{Diamond2013},
there are broadly speaking two kinds of inhibitory processes.
One is \emph{response inhibition},
which allows us to override automatically executed motor commands.
The purest (i.e. least confounded by other factors) measures of this kind of control
are the Go/No-Go \citep{Donders1868} and Stop-signal \citep{Lappin1966} tasks.
Both of these require participants to quickly give a response
associated with a given stimulus,
except on trials where a no-go or stop signal is presented.
The other inhibitory process is \emph{cognitive inhibition},
which inhibits and filters out irrelevant or unwanted
mental representations and information.
Of particular relevance here is \emph{semantic} inhibitory control,
a subset of cognitive inhibition measured using the Hayling task \citep{Burgess1997}.
In the non-clinical version of this task \citep{Markovits2004},
participants are primed with incomplete sentences
(e.g. ``The captain wanted to stay with the sinking $\rule{1cm}{0.15mm}$``),
and then shown a single word and asked to indicate if it is an appropriate word
with which to finish the sentence.
Crucially, on lure trials the probe was a pseudo-word
similar to a real word primed by the sentence ---
i.e. ``shifp'' when ``ship'' was expected.
This requires that participants inhibit their automatically primed
semantic knowledge to correctly reject this word.

\citet{Bright} presented participants with the triad task from Experiment 3,
which placed associative and structured knowledge in conflict.
They also collected measures of both response inhibition (the Stop-signal task)
and semantic inhibition (the Hayling task).
Their results showed that the effect of associative knowledge
--- fewer correct responses when associative knowledge cued the foil ---
was more pronounced for participants who performed poorly on the Hayling task
(responded that the lure pseudo-words were appropriate),
but was not related to individual differences in
performance on a Stop-signal task.
Therefore, they concluded that foil responses on this task
are the result of a failure to inhibit
inappropriate semantic knowledge in favour of structured knowledge,
rather than a failure to inhibit the inappropriate response itself.

Applying this logic to Experiments 1 and 2,
I would predict that if foil responses on these tasks
are the result of conflict between representations in working memory
like foil responses in Experiment 3 and 4, and in \citegap{Bright}{'s} experiment,
participants lacking in semantic inhibitory control
should be more likely to err in this way.
Conversely, if these foil responses
are instead the result of a failure to inhibit
the response automatically cued by the visual cues,
it should be participants with poorer response inhibition
that err in this way instead.
Clearly, further research is needed to resolve this question.

Beyond this question, however, the experiments reported in both
Chapters 3 and 4 have implications for theories of induction more broadly.
First, these results are consistent with \citegap{Bright2014a}{'s} argument
that theories of induction that draw on just one kind of knowledge are incomplete.
Instead, it seems that induction can be driven by multiple forms of information.
\citet{Bright2014a} showed that different sources of information are used
depending both on individual differences (semantic inhibitory control)
and on situational factors (the presence of secondary load).
The current results showed that multiple kinds of information
can drive reasoning even within a single trial, leading to conflict.

Formal modelling plays an important role in research on induction
\citep[see, e.g.,][]{Sloman1993,Osherson1990,Kemp2009,Hawkins2015}.
These results, however, are challenging to model using existing frameworks.
A natural next step, following on from \citegap{Bright2014a}{'s} hybrid account,
would be to attempt to model these time course data
using a hybrid of existing models based on associative and structured knowledge.

This could potentially be achieved, for instance,
by a change-point model where reasoning is driven by
one kind of information before time $t$,
and by another model afterwards.
Another potentially interesting avenue here is
the CLARION cognitive architecture \citep[e.g.][]{Sun1995,Sun2006},
which combines implicit similarity-driven
and explicit rule-based operations.
\citet{Sun2006}, for instance, use this architecture
to model performance on a number of inductive reasoning problems.
However,  at present this model does not account for
the time course data presented here,
and so further work would be needed
to model the full temporal dynamics of reasoning.

%% \aside{I'm not totally sure this next paragraph
%%   should be here at all.}
%% Another question that arises is whether or not
%% participants always experience conflict here.
%% While I have said much about the strengths of the mouse tracking paradigm,
%% one obvious weakness is that it only reflects ongoing cognitive processes
%% once participants actually begin to move the mouse.
%% There is unavoidably%
%% \footnote{
%%   Unavoidable unless one uses the approach introduced by \citet{Scherbaum2010}
%%   of only revealing the stimuli once participants begin moving the mouse,
%%   which I found to be unsuitable to reasoning problems.
%% }
%% an interval between participants seeing the trial stimuli
%% and beginning to move the mouse.
%% Therefore, while participants do not initially
%% move the cursor towards the foil on conflict trials,
%% it is not clear if this is because they are
%% genuinely not drawn towards the foil on such trials,
%% or because they are able to inhibit this attraction
%% before initiating their movements.
%% One might expect movement initiation time to be a useful measure here.
%% Unfortunately, the results here were also unclear:
%% there were no effects on initiation time in Experiments 1 and 2,
%% and an effect in Experiment 3, but not 4,
%% despite there being no changes between the experiments
%% that should influence this.
%% However, there is perhaps good reason for movement initiation times
%% to be a confusing index of conflict in cognition.
%% As discussed in Chapter 2,
%% in many tasks, participants are placed under conflict
%% by changing only the identity of the foil option
%% to something towards which participants will be drawn.
%% Therefore, on these conflict trials,
%% participants are drawn towards moving the mouse by two factors
%% --- the pull of the correct option, and the pull of the foil,
%% rather than just the pull of the correct response.
%% Coupled with the pre-emptive inhibition of conflict mentioned above,
%% this may account for the inconsistent initiation time results.
%% However, this leaves unanswered the question of
%% whether or not participants experience conflict on all trials,
%% and so it is a question I must leave open for the time being.






\subsection{Discussion}

This experiment placed perceptual similarity
and conceptual knowledge in conflict
in an inductive reasoning task.
My first goal in doing so was to test the prediction,
based on \citegap{Bright2014a}{'s} hybrid theory,
that adults' inferences would be influenced by perceptual cues,
as well as conceptual knowledge.
This prediction was confirmed by participants' responses:
when conceptual knowledge did not conflict with perceptual similarity,
participants selected the conceptually-cued response on almost every trial.
When the foil species looked similar to the base, however,
participants chose the foil 40\% of the time.
Additionally, this effect was found for the vast majority of participants,
and so is not the result of a few participants who are
inappropriately swayed by these visual cues.
When participants did give the correct response,
it appears that they experienced conflict
when perceptual cues supported the foil response,
taking significantly longer to respond,
and being more likely to initially move towards
the perceptually-cued foil option.

The mouse tracking paradigm makes it possible
to go beyond this finding, however,
and work towards a description of
how these two sources of information interacted during reasoning.
In Chapter 1, I raised two possibilities:
participants may either selectively use one or other cue,
or they may use perceptual cues by default,
but reject them in favour of conceptual knowledge
when they realise those cues to be inappropriate.
Consistent with both possibilities,
on control trials, where the base and the foil did not look alike,
participants initially moved towards the foil 21\% of the time.
On conflict trials, where they did look alike,
participants initially moved towards the foil 57\% of the time.
Furthermore, these movements %towards the foil under conflict
were initiated faster than movements towards the correct option in the same condition,
suggesting either
that conceptual knowledge takes longer to utilise than perceptual cues,
or that perceptual cues must be inhibited on these trials.

These two possibilities can be distinguished by looking to
what participants do when they have initially moved towards
a perceptually-cued foil.
In this situation, participants changed direction on 45\% of trials
to select the correct option instead.
By comparison, participants who initially moved towards the correct option under conflict
only changed direction to select the foil on 19\% of trials.
Therefore, it appears that, at least some of the time,
reasoning on the basis of conceptual knowledge required
the inhibition of the response based perceptual cues.
Together, these results suggest that perceptual cues
are used by default on this task.
By this view,  participants can either
initiate an early movement, driven by these cues (57\% of the time),
and subsequently either inhibit this response (45\% of the time),
or follow through with it and select the foil species.
Alternatively they can override these cues
before they move the cursor (43\% of the time),
and move directly towards the correct option.

Analysis of the time course data also appears to support this interpretation.
In Figure~\ref{fig:exp1_side_timecourse} we see that
on conflict trials, participants were initially (300 to 1,340 msec)
more likely to be on the foil side of the screen than the correct side.
By 1,940 msec, however, this trend had reversed,
and participants were instead more likely to be on the side of the correct option.

An interesting question, which this experiment does not answer,
is what factors dictate whether or not participants
moving towards the foil ultimately select this option,
or instead override this initial movement and select the correct species.
  %% In particular, a lot of recent Thompson/Pennycook work
  %% has looked at determinants of reflection from a DI perspective.
  %% Their account talks about metacognition, and Feeling of Rightness,
  %% in contrast to De Ney's intuitive conflict detection idea.
  %% For now, I'll just cite that work here, on the basis that
  %% the next draft of Ch1 will review it.}
This issue has come under increased scrutiny lately in the dual process literature
\citep[see, for instance,][]{Thompson2014a,Thompson2011,DeNeys2012,Pennycook2015},
where the focus has been on whether participants
reflect on their intuitive responses.
In the induction literature,
\citet{Gelman2013c} report a series of experiments with children
using the triad task.
They show that children are more likely to
use conceptual knowledge, rather than perceptual similarity,
when the conceptual categories used differed at a high, ontological level,
(animals versus robots),
than when they differed at a lower level (kinds of dogs, or creatures categorised
according to their ratio of fingers to chest buttons,
as used by \citealp{Sloutsky2007}).
They also demonstrated that children are more reliant
on conceptual knowledge when the properties under consideration
are meaningfully related to the different categories,
for instance animals being warm blooded, or robots containing batteries.
Both of these effects make sense from a normative perspective:
categories which are more conceptually distinct
provide a better basis for induction \citep{Rosch1976},
and certain kinds of category are conducive to
the projection of certain kinds of property,
for instance biological properties within biological categories
\citep{Heit1994,Shipley1993,Shafto2007}.
We would expect these same factors to influence adults' inferences.
However, it is not clear at what point in the reasoning process
such variables are important:
they may prompt reasoners to attempt
to draw on conceptual knowledge in the first place,
or they may cause them to be more likely to inhibit
their initial inappropriate perceptually-driven responses.

In Experiment 2, I presented participants with a version of this task
using artificial categories, specifically, a kind of animal, and a kind of robot.
I also manipulated, between participants,
the nature of the kind of properties being considered,
in order to investigate the effect this has on participants' reasoning.

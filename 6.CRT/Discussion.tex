
\section{Discussion}

These results are broadly consistent with
a default-interventionist dual process theory
\citep{Evans2006,Kahneman2005}.
On problems with an incorrect but intuitively appealing heuristic response,
this response was given more quickly,
and with less evidence of conflict,
than the correct response.
Participants began to systematically move the mouse cursor
to the region of the heuristic response option within approximately 5 seconds,
compared to 10 seconds for movements to the correct response option,
and this trend was evident both when analysing all trials,
and trials in which the response in question was given.
This also appears to be true of both biased and unbiased participants. 

When participants did give the correct response on these conflict problems,
they spent more time in the region of the heuristic response option
than either of the foil options before doing so ---
a finding consistent with default-interventionist,
parallel-competitive, and intuitive logic accounts,
suggesting that these participants considered the heuristic response
before they reached the correct one.
This finding is also consistent with modelling work
\citep{Bockenholt2012,Campitelli2013},
and individual differences studies \citep{Liberali2012}
which have shown that inhibition of the heuristic response
is an important predictor of accuracy on the CRT.
However, contrary to the prediction made from
a parallel-competitive dual process theory
\citep{Sloman1996,Sloman2014}
or by the intuitive logic account
\citep{DeNeys2014a,DeNeys2012,Handley2015}
on trials where the heuristic response was given
participants were no more likely to place the cursor
in the region of the correct response option than either foil option.

These results also have implications for the logical intuitions theory
\citep{DeNeys2012,DeNeys2014a}.
First, a number of previous studies using simpler reasoning tasks
have found that heuristic responses to conflict problems take longer
than correct responses to no-conflict problems,
despite both being cued by Type 1 processes
\citep{DeNeys2008,Stupple2008}.
To my knowledge, the current study is the first to report response times
for conflict and no-conflict versions of the CRT,
and although this analysis was not the main focus of the this experiment,
I found no such effect.
In fact, when analysing participants speed of movement
to the response option they ultimately selected, a more sensitive measure,
I found the opposite effect,
with participants faster to move to the heuristic option under conflict
than the correct option for no-conflict problems
on trials where these responses were given.
All of these findings were true for both
participants who gave the heuristic response to most conflict problems,
and those who did not.
Thus, unlike a number of studies using simpler reasoning problems,
I found no evidence that participants were slower
to give intuitively-cued responses which were wrong
than intuitively-cued responses which were right.

Secondly, as discussed above, I found no evidence of
an attraction towards the correct response option
on conflict problems where the heuristic response was given.
This suggests that Type 1 processes did not simultaneously
cue both responses on such trials.
This result is not, however, totally inconsistent with the logical intuitions theory.
Previously, I differentiated between
a dual process conflict monitoring account \citep{DeNeys2008,DeNeys2008a},
that proposes that we detect when our reasoning is biased,
and a fully-fledged intuitive logic theory \citep{DeNeys2012,DeNeys2014a,Handley2015},
where this conflict detection is the result
of Type 1 processes simultaneously cuing
both the correct and the heuristic response.
These results would appear to contradict the latter account,
%% \citep{DeNeys2012,Handley2015,Pennycook2015,DeNeys2014a}
whereby Type 1 processes cue both the heuristic response (``10p'')
and the correct response (``5p'') at the same time
for the bat-and-ball problem and other CRT items.
They do not, however, rule out the possibility that
participants experience uncertainty \citep{DeNeys2013a},
or a feeling of wrongness \citep{Gangemi2015}
while solving these conflict problems.
If this is the case, further work is needed to reveal how this feeling comes about.

At this point, I would like to note again that,
\citet{DeNeys2013a} and \citep{Gangemi2015} notwithstanding,
previous evidence for the intuitive logic theory has come from simpler experiments,
such as simple syllogistic reasoning \citep{Morsanyi2012}
and forced-choice base rate neglect \citep{DeNeys2008} paradigms.
The operations required to reach the correct answer to these CRT problems
are considerably more complex than those needed to evaluate a simple syllogism,
or apply basic statistical principles.
Therefore, while I do not find evidence that Type 1 processes
automatically generate correct responses on the CRT,
this does not rule out the possibility that they can
generate correct responses on these simpler tasks.
For instance, it has been demonstrated that participants
report ``liking'' syllogisms which are logically valid
more than those which are invalid,
even when not asked to evaluate their logical status \citep{Morsanyi2012},
but also that this effect only holds for simpler logical forms
\citep[see also \citealp{Handley2015}]{Klauer2013}.
Indeed, \citet{DeNeys2012},
when proposing the intuitive logic account
raised the possibility that it
may not apply to all problems.
%% With this in mind, it is worth considering if
%% the reductions in confidence for conflict problems reported by
%% \citet{DeNeys2013} and \citet{Gangemi2015} can be explained
%% without recourse to logical intuitions.
%% One possibility is that participants were engaged in \emph{rationalisation}
%% \citep{Ball2003,Evans2006,Pennycook2015}.
%% That is, while Type 1 processes cued initial responses,
%% participants in both conditions may engage shallow Type 2 processes
%% to attempt to verify them.
%% For no-conflict problems, participants can easily verify
%% that their intuitive response is correct,
%% and so give it with confidence.
%% For conflict problems, on the other hand,
%% their initial responses would be incorrect,
%% and so they would not be able to validate them.
%% Some participants would give these unvalidated responses,
%% but lack confidence in doing so,
%% while others would engage further Type 2 processes to produce the correct response.
%% Crucially, such an account would not require that
%% participants implicitly detect the error in their initial responses.
%% This account should also be testable in future.
%% If the reduction in confidence is the product of an implicit detection of conflict,
%% it should occur early in the reasoning processes.
%% Alternatively, if it is the product of
%% an inability to explicitly validate an intuitive response,
%% the reduction in confidence should occur later,
%% after these shallow Type 2 processes have been engaged.
%% I would stress at this point, however,
%% that I do not dispute the validity of the intuitive logic theory more broadly here,
%% but rather question if it applies equally to complex logical tasks as to simple ones.

One might argue that the absence of evidence for either
the parallel-competitive or intuitive logical theories here
do not reflect evidence against these accounts,
but rather the inability of this paradigm to reveal
the effects predicted  by these accounts.
It may be the case, according to this line of reasoning,
that participants are drawn towards the correct option
on trials where they give the heuristic response,
or that participants are more conflicted
when their heuristic responses are wrong than when they are right,
but that I was unable to detect these mental states using this new paradigm.
While I cannot completely rule out this possibility,
I believe two factors go against such an interpretation.
First, this paradigm does reveal effects,
such as attraction towards the heuristic option before giving the correct response,
consistent with the default-interventionist model.
Second, for the two comparisons above, rather than finding no effect,
I found significant effects in the opposite direction
to those predicted by parallel-competitive and intuitive logic accounts.

Additionally, there is extensive evidence that
even subtle, implicit cognitive processes
influence motor output in detectable ways
\citep{Tucker2004,Xiao2014,Miles2010,Bargh2006}.
Therefore, if participants do experience conflict,
but this conflict does not influence their motor output,
then this raises the question of
what  mechanism produces this conflict
while not influencing motor output.
I return to this issue in Chapter 7.
Finally, I would note again that these results
should be interpreted as
constraining the intuitive logic account,
rather than falsifying it.

Of course, all of the above assumes a dual process interpretation of the CRT,
as most treatments of the task do.
Even in accounts which focus instead on dispositional factors
\citep{Campitelli2013,Campitelli2010a},
it is acknowledged that responding correctly typically requires
the inhibition of the heuristic response.
While I am unaware of any accounts of the CRT
which do not rely on such an inhibition,
I cannot rule out the possibility of such explanations being offered in future.
The current results, however, provide an additional constraint to such accounts,
in that they should predict not only observed choices,
but also the patterns at the process level reported here.

As a side note, it may be noted that it is unusual
to present the CRT as a multiple-choice test,
and that this may affect the processes engaged during this experiment.
However, multiple-choice versions for the test have been previously reported
by \citet{morsanyi2014mathematical}, \citet[][Experiment 3]{Primi2015},
and \citet[][Experiment 2]{Gangemi2015},
without any clear effect on participants’ responses.


Finally, since its introduction in 2005, the CRT has been hugely popular
as a measure of individual differences in thinking,
despite only limited evidence as to what underlies performance on the task.
These results go some way towards filling this gap,
and suggest that responding correctly
does require the activation of otherwise dormant Type 2 processes
to override incorrect intuitions.
Future work might address the relationship between
conflict on this task and individual differences.
\citet{Stanovich2008} proposed that normative decision making requires
(1) awareness of the limitations of intuition;
(2) desire to overcome those limitations;
(3) inhibition of the intuitive response and
(4) ability to generate the correct response.
Each of these requirements is a distinct reason for
failure to produce the correct response on the CRT,
and each should produce a distinctive pattern in mouse cursor movement data.

To conclude, I recorded participants' mouse cursor movements
over a considerable period of time
while they reasoned about CRT problems.
Trends in these movements were consistent with
a default-interventionist dual process theory of reasoning,
where participants are initially drawn towards heuristic responses only,
but in some cases engage further effortful processing to find correct solutions.
I did not find evidence that participants were
drawn to correct responses on trials where these responses were not actually given,
inconsistent with a parallel-competitive dual process account.
Finally, contrary to previous work using simpler reasoning tasks,
and confidence ratings collected on the CRT,
I found no evidence that participants were conflicted
when giving incorrect heuristic responses. 



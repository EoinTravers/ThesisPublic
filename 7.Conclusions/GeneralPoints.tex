
\section{Conflict more broadly}\label{subsec:ch7-broadly}

Aside from the implications of my results
for theories of reasoning,
there is much in this thesis that may be of interest
to those interested in conflict in cognition more broadly.
In this section, I discuss a number of these broader issues.

\subsection{Changes of Mind in Cognition}

In Chapters 1 and 2 (see also Appendix~\ref{appendix:mouse-studies}),
I discussed previous applications of the mouse tracking paradigm.
Most of these studies to date,
studying simple cognitive and perceptual tasks,
have revealed continuous attraction effects,
as participants are partially and simultaneously drawn towards two responses.
However, a number of studies,
mostly of more complex, high-level cognition
\citep[e.g.][]{Dale2011,Tomlinson2013,McKinstry2008,Barca2015}
instead show \emph{reversals}, or \emph{changes of mind}
as participants move first towards one response option,
and then towards the other.
In Experiments 1 to 5 of this thesis, similarly,
I found that my high-level reasoning tasks yielded these changes of mind,
and little evidence of continuous attraction
(see Appendix~\ref{appendix:reversals}).
At present, however, it is not yet clear why
conflict works one way (continuous attraction) for simple tasks,
and another (changes of mind) for more complex ones.

Above, I proposed a simple framework
for thinking about conflict in reasoning.
An important idea here was that,
according to many accounts
\citep{Evans1984,Evans2006,Johnson-Laird1983,Legrenzi1993,Mynatt1993,Baddeley2003},
only one state of the world can be represented in working memory at a time:
it can believe that the correct response is A, or that it is B,
or consider each in turn,
or assign explicit probabilities to each possibility,
but it cannot simultaneously represent
a world where the correct response is A
\emph{and} a world where the correct response is B.
Such fuzzy, graded, partially-overlapping representations, however,
are fundamental to other theories of cognitive function,
including those based on neural networks
\citep{McClelland2014,Rogers2004,McClelland1986a,Hinton2014,Kruschke1992},
sequential sampling models
\citep{Leite2010,Ratcliff2010,Ratcliff1978,Townsend1995,Usher2001},
and dynamical systems \citep{VanGelder1998,Port1995,Spivey2007,Freeman2011a}.
More broadly, there has been philosophical debate as to whether
mental representations in general are discrete and symbolic
\citep{Newell1972,Johnson-Laird1983,Johnson-Laird1991,Dietrich2003},
or continuous, graded, and fuzzy
\citep{Huette2012,Spivey2007,Beer1995,Port1995}.

I propose that continuous graded representations,
and continuous attraction effects,
are the hallmark of processes that do not require
sustained representations in working memory,
or, in other words, of Type 1 processes.
Due to the unique nature of the symbolic,
domain-general processing done in working memory, however,
processes that rely on this (that is, Type 2 processes)
operate in an all-or-nothing fashion:
we ultimately decide that either A or B is the correct response.
Of course, it is possible to explicitly consider uncertainty in the world,
but doing so requires a great deal of cognitive effort
\citep{Murphy2012,Malt1995,Tversky1982}.
Therefore, there is a \emph{symbolic bottleneck} in working memory ---
representations that were continuous and graded
are forced into discrete symbolic states
when they are passed to working memory.
\citet{Dale2005a} provide a formal model
for how continuous and symbolic representations could map onto each other,
using the mathematical framework of \emph{symbolic dynamics}.

This distinction corresponds well with
the claim by \citet{Oaksford2010,Oaksford2012} that
the representations used by Type 1 processes are probabilistic in nature,
while those used by Type 2 processes are approximately symbolic.
It also dovetails with recent experiments
showing that probabilistic tasks that participants normally struggle with
\citep[e.g.][]{Kahneman1979,Murphy2012}
can be solved easily when presented in a way
that allows the inferences to be performed
by perceptual, motor, and low-level cognitive systems
\citep{Chen2013,Glaser2012,Wu2009,Trommershauser2008a,
  Trommershauser2008,Trommershauser2006,Kording2004}
--- or, in other words, by Type 1 processes.
In a similar vein, there is an extensive literature showing
that the human perceptual system routinely solves complex Bayesian inference problems
\citep{Wolpert2005,Kersten2004,Roach2006}
considerably more complex than those
even highly numerate individuals struggle to solve explicitly
\citep[e.g.][]{McNair2013,Barbey2007}.

I predict that changes of mind
should emerge in mouse tracking studies of processes
where Type 2 processes,
involving maintaining and manipulating information in working memory,
are required to produce at least one of the possible responses.
That being said, this is not the only situation
that can give rise to changes of mind.
Changes of mind have been demonstrated, for instance,
when participants move initially
to categorise pseudo-words as real words \citep{Barca2015},
to categorise short-haired women or long-haired  men
as members of the opposite sex
\citep[in contexts where such styles are rare;][]{Freeman2014a},
to respond that scalar implicatures (``some elephants are grey'')
are true, before deciding them to be false \citep{Tomlinson2013},
to ignore the ``not'' in a negated sentence \citep{Dale2011},
as well as when switching between tasks \citep{Hindy2008}.
Of course, while arguments could be made that some of these tasks
can be interpreted from a dual process perspective,
it is also possible that these changes of mind arose from
the interaction of two simple processes, activated one after another,
or even from a single process accompanied by a shift in attention.
My claim therefore is not that changes of mind are
diagnostic of dual processes phenomena
and the engagement of working memory,
but that continuous attraction, and the absence of changes of mind,
are indicative of tasks based only on Type 1 processes.


\subsection{Conflict in the Hand/Conflict in the Mind}

One possible criticism of my findings is that
cognitive conflict may not always have
a measurable effect on moues cursor trajectories.
For instance, the failure of these mouse tracking data
to support the intuitive logic account may simply mean that
although participants do implicitly experience conflict,
this conflict does not have a detectable influence on their motor output.

However, a great deal of previous work has demonstrated that
even subtle, implicit cognitive processes
produce measurable effects in participants' motor output.
Simply seeing objects, for instance, primes us to perform
the actions afforded by them
\citep[their \emph{microaffordances};][]{Ellis2000,Tucker1998,Tucker2004}.
Subtle priming influences have also been demonstrated
using the mouse tracking paradigm,
including on simple judgement tasks \citep{Xiao2014,Finkbeiner2008},
spatial congruency effects \citep{Tower-Richardi2012}
and even temporal congruency effects \citep[i.e. the past is on the left;][]{Miles2010}.
These effects show that specific, actionable information
passes from the simple cognitive and perceptual processes to the motor system.

The question remains, however,
as to why the current results are not in accord with
the predictions of the intuitive logic theory.
From previous work, there is much evidence
that people are sensitive to conflict
when giving heuristic responses during reasoning.
These data are mostly epiphenomenal,
in that they reflect the by-products of conflict in reasoning,
such as ACC activation \citep{DeNeys2008a},
the galvanic skin response \citep{DeNeys2008},
affective appraisal \citep{Morsanyi2012},
or metacognitive confidence \citep{DeNeys2011b}.
Given the diversity of evidence for the intuitive logic model, however
\citep[see][for a review]{DeNeys2012},
I do not argue that the two null effects reported here
serve to falsify the account.
Why, then, when so much information passes continuously
from cognitive and perceptual process to the motor system,
does no evidence of the kind of conflict in reasoning
predicted by an intuitive logic account
feed into cursor trajectories on these tasks?

I cannot answer this question based on the evidence at hand.
However, I would suggest a few possibilities.
First, as I noted in Chapter 1,
it may be the case that participants are implicitly aware
that their responses and normative principles,
as proposed by what I labelled
the dual process conflict monitoring theory \citep{DeNeys2008,DeNeys2008a},
but contrary to the intuitive logic accounts \citep{DeNeys2012,DeNeys2014a},
this is not because Type 1 processes are attempting to cue the correct response.
At this point, however, it is not clear what other mechanisms
could underlie these conflict detection effects.

Second, it could be that the kind of implicit,
cognitive monitoring invoked
by the intuitive logic theory \citep{DeNeys2012},
as well as by conflict monitoring accounts more broadly \citep{DeNeys2008,Botvinick2004a}
--- what \citet{Shea2014c} label \emph{Type 1} metacognition --- 
does not feed information directly to the motor system,
whereas non-metacognitive implicit processes of the kind discussed above do.
Again, this raises further questions about the mechanisms
that underlie logical intuition.

Thirdly, it is possible that
conflict detection in reasoning,
when it does occur,
is not an entirely Type 1 processes.
Rather, it may be mediated in some way by Type 2 processes
--- although not impinging enough on working memory
that it is affected by secondary load manipulations \citep{Franssens2009},
or available to conscious introspection \citep{DeNeys2008}.
Above, I argued that Type 2 processes dependent on working memory
impose a symbolic bottleneck on cognition.
If this conflict detection relies on such processes,
it may be that this bottleneck prevents it
from reaching the motor system.
Again, while I do not believe
these questions can be resolved using
the current data alone,
this discussion highlights some of the value
of analysing more than simply participants' ultimate responses.


\subsection{Mouse Tracking, Fast and Slow}

Since its inception \citep{Spivey2005},
the mouse tracking paradigm has for the most part
been applied to the study of simple processes
that unfold over at most two or three seconds.
As discussed above, these tasks have largely
yielded continuous attraction effects,
as participants' cursor trajectories
curve slightly more on conflict trials than no-conflict trials.
Here, I have applied this method to tasks that
typically take considerably longer than this to perform.
Previous experiments, for example using
the base rate neglect task as used here,
minus the time pressure constraint,
typically find response times of \tildetext18 seconds on conflict trials
(and \tildetext14 seconds on control trials;
\citealp{DeNeys2008,DeNeys2009a,Franssens2009}).
Similarly, while no work has collected response times
for the inductive triad task pitting associative against structured knowledge \citep{Bright},
I would assume that participants rarely responded
within two or three seconds when completing the task normally.
However, time pressure was necessary
for the mouse tracking paradigm to work:
in pilot studies where
participants were not placed under time pressure on these tasks,
they did not move the mouse until they had made their decision,
and so the mouse data revealed little.
Fortunately, the triad task pitting
perceptual cues against perceptual knowledge (Chapter 3),
being adapted from a task used with children,
was relatively easy for participants to perform,
and so they responded quickly, and moved the mouse while deciding,
without the need for extrinsic time pressure.

As should be obvious to readers familiar with speeded response paradigms,
requiring participants to reason under time pressure
can influence the nature of the reasoning they engage in.
In particular, in the dual process literature,
participants are often required to respond quickly
in order to record their \emph{Type 1} responses
\citep[e.g.][]{Villejoubert2009,Markovits2004,DeNeys2006a,Thompson2011}.
Similarly, \citet[Experiment 1]{Bright2014a} demonstrated that
the influence of structured knowledge in induction,
but not of associative knowledge,
was reduced when participants were placed under time pressure.
Therefore, we can expect my data on these tasks
to be less influenced by Type 2 processes,
or structured knowledge,
than would be the case if participants were not under time pressure.
Fortunately, this does not affect my conclusions.
In the induction tasks (Experiments 3 and 4),
placing participants under time pressure
may have rendered them less likely to draw on structured knowledge,
and more likely to draw on associative knowledge.
This possibly explains the poor structured knowledge
displayed by participants in Experiment 3,
as well as the fact that even on control trials
participants did not universally give the structurally-cued response.
However, while perhaps amplified by it,
the key finding here can not be
caused by time pressure alone:
both associative and structured knowledge
are drawn on in induction.

In the same vein,
when descriptions and base rates conflict in Experiment 5
participants only gave the base rate-cued response 20\% of the time,
considerably less than the norm in such experiments (see Table~\ref{tab:previous_baserate_studies}).
Similarly, in contrast to previous work,
manipulating the base rate
did not influence participants' response times
when they gave the description-cued response,
again contrary to previous findings
\citep[e.g.][see again Table~\ref{tab:previous_baserate_studies}]{DeNeys2008}.
While all of this is consistent with a default-interventionist account,
it differs from previous results in favour of the intuitive logic account.
%% \aside{I'm not 100\% on the next paragraph.}
It is tempting to ascribe these differences to
the speed of participants' responses in Experiment 5,
which were even faster than those in
previous speeded tests of the accounts \citep[Experiment 2]{DeNeys2008a,Pennycook2014}.
However, a core aspect of the intuitive logic theory
is that this conflict occurs because
both responses are cued by fast, automatic Type 1 processes,
rather than requiring the engagement of slower Type 2 processes,
as would be the case in the parallel-competitive theory \citep{Sloman1996}.

A few possibilities are consistent with this pattern of results.
One is that, as discussed above, conflict detection and the cuing of the correct response
do require the engagement of Type 2 processes, consistent with a parallel-competitive account.
An alternative is that while Type 1 processes do cue the correct response,
doing so still requires more time than is available here.
However, \citet[see also \citealp{Pennycook2014,Handley2011}]{Handley2015}
have recently argued that speed of processing does not provide a good means
of differentiating between Type 1 and Type 2 processes,
as  some Type 1 processes can operate relatively slowly,
and some Type 2 processes relatively quickly.
For this reason, it is difficult to draw strong conclusions here.

%% Aidan:
%%     this last sentence is the contentious one. YOu could soften things by
%%     merely saying that as intuitive logic approaches claim that people are
%%     at some level aware f conflict between the output of competing Type 1
%%     processes, it isn't clear how that approach can explain the absence of
%%     effects in terms of participants' unusually fast reaction times.


\subsection{Future directions}

At this point, I hope to have demonstrated
some of the value of the mouse tracking paradigm
adopted in this thesis.
I will conclude this thesis
by discussing what role mouse tracking may play in future research.

\subsubsection{The Development of Reasoning}

Although these experiments focused on adults' reasoning,
children's inductive reasoning \citep{Gelman2013c,Bright2014,Sloutsky2007},
as well as their reasoning more broadly \citep{OConnor2012,Steegen2012,Handley2004},
remains an active topic of research.
In particular, Experiments 1 and 2,
pitting perceptual similarity against conceptual knowledge,
took their inspiration from developmental studies
seeking to reveal which of these sources of information four-year-old children rely on
\citep{Gelman2013c,Gelman1986,Sloutsky2007}.
In this literature, it is typically taken for granted
that adult reasoning is based on conceptual knowledge,
usually on the basis of adult control groups
that complete conflict problems,
and rely on conceptual knowledge almost all of the time
\citep[e.g.][]{Gelman2013c}.
Recall that while my (adult) participants did indeed
give the conceptually-cued response the majority of the time on conflict trials,
they did so significantly more often on no-conflict control trials,
showing that they are also driven by perceptual cues.

An interesting next step here would be to
adapt the mouse tracking paradigm to collect data with children.
Of course, it remains to be seen if
young children are capable of projecting their unfolding cognitive states
onto the position of the mouse cursor
in the way that came naturally to adult participants
extremely adept at interacting with a computer in this way.
One possibility we are currently exploring
is the viability of low-cost motion tracking systems
to tracking children's finger movements
in pointing versions of this task.
Such finger tracking has been used in the past
in analyses almost identical to those used in mouse tracking studies
\citep[e.g.][]{Song2008,Song2008a,Song2006},
and may provide a new window into children's reasoning.


\subsubsection{Individual Differences}\label{subsec:ch7-ids}

There are, of course, individual differences in reasoning
\citep[see, e.g.,][]{Stanovich2000,DeNeys2005,Frederick2005,Feeney2007a}.
Variability in participants' responses has been attributed to
individual differences in general intelligence \citep{Stanovich2000},
working memory capacity \citep{DeNeys2005},
personality traits \citep{Cokely2012,Cacioppo1982}
including cognitive reflection \citep{Frederick2005,Stanovich2009a,Stanovich2009},
inhibitory control \citep{Bright,Markovits2004}
and even differences in circadian cycles
\citep[i.e. \emph{morning people} are more rational in the morning;][]{Wieth2011}.

These individual differences  were not the focus of this thesis,
and the designs used rarely provided enough data from each participant
to reliably estimate individual differences parameters.
However, the novel data generated by the mouse tracking paradigm used here
may prove useful in future individual differences work.
In other words, it would be useful for future work
to investigate individual differences in not only participants' responses,
but also the more subtle analyses reported here.
One particular candidate is the transition probability analysis.
It would be informative, for instance,
to discover what factors predict whether participants will
initially move towards the foil response option (parameter $1 - \alpha$),
whether they select a correct response after moving towards it (parameter $\beta$),
and whether they will override initial movements
towards a heuristically-appealing foil (parameter $\gamma$).
Although not directly used in this way here,
this type of analysis would also be useful
in evaluating frameworks for individual differences in reasoning
discussed in Chapter 1.

In Chapter 1, I discussed three such frameworks
for individual differences in reasoning.
%% \citep{Stanovich2008,DeNeys2013,Pennycook2015}.
\citet{Stanovich2008} constructed their framework
by correlating individual differences in variety of predictive factors
(fluid intelligence, formal knowledge, inhibitory control, thinking dispositions)
with differences  in participants' \emph{responses} on reasoning tasks.
\citet{DeNeys2013} go somewhat further by considering the question of
\emph{when} these individual differences arise.
The mouse tracking paradigm,
and analyses of temporal dynamics more broadly,
may provide a new means of investigating these ``when'' questions.
Lastly, \citet{Pennycook2015} synthesise the theory
and findings in support of the intuitive logic theory,
discussed in Chapters 1, 5, and 6,
many of which involve subtle behavioural and biological measures.
Mouse tracking data, however, could go further than this.
In particular, along with participants' responses and response latencies,
the paradigm yields individual differences in the cursor parameters
$\alpha$, $\beta$, and $\gamma$, above.
Clearly, much further work is required
to establish a coherent account of how individual differences,
and differences in task characteristics,
relate to variance between participants in these parameters.


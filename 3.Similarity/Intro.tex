
\section{Introduction}

In Chapter 1, I discussed how people can
draw on different sources of information in induction.
In this chapter, I report a pair of experiments in which
two such sources --- perceptual similarity,
and conceptual knowledge in the form of category membership ---
are placed into conflict.
This was done using a mouse tracking version of
\citegap{Gelman1986}{'s} inductive triad task.
In Experiment 1, participants learned a biological property of
different species in the natural world, and were asked to
project each property to one of two species,
one of which belonged to the same taxonomic group as the base species.
In the conflict condition, but not the control condition,
the base and the foil species looked alike,
and so perceptual cues conflicted with conceptual knowledge.
In Experiment 2, participants completed a similar task,
but using artificial categories, learned at the start of the experiment.
I also manipulated the nature of the properties
participants reasoned about in Experiment 2, between participants,
to reveal how this modulated participants' inferences.

As discussed in Chapter 1,
theories of inductive reasoning can be organised into two groups.
One class of theory relies on conceptual knowledge
about the categories to which different entities belong,
and the relations between these categories.
Category membership is perhaps the most fundamental 
such form of conceptual knowledge that we can use as the basis for induction.
According to contemporary accounts of categorisation
\citep[see][]{Murphy2004,Rosch1988},
a category consists of a set of entities that have attributes in common.
Therefore, if we know the category to which something belongs,
we may believe that it has attributes
that we know are found in other members of that category
\citep{Murphy2004,Osherson1990}.
That adults can reason in this way has never been in question,
and indeed this idea is central to the very notion of categorisation
\citep{Mill1856,Gelman1986}.
This principle applies equally to inferences about people,
where it is better known as \emph{stereotyping}
\citep{Greenwald1995,Oakes1994}.
% The evidence that we do use knowledge about categories when reasoning
% is considerable, and indeed this should be self-evident,
% and so will not be reviewed in detail here.
% Few results in literature on inductive reasoning in adults
% \citep[see][for reviews]{Feeney2007,Hayes2010}, for example,
% can be accounted for without recourse to category-based inference.
Strikingly, people rely on categorical information
even when it is disadvantageous to do so.
Murphy and Ross \citep{Murphy2012,Murphy2010,Malt1995,Murphy1994}
showed, for instance, that when it is not certain
to which category an entity belongs,
participants nevertheless reason on the basis
of it belonging to the most likely category.
\citet{Mishra2010} also report a \emph{border bias} phenomena:
distant locations in the same state (for instance New York City, and Buffalo, New York) 
are perceived as physically closer, and more likely to share properties,
than closer locations divided by state lines (New York City and Princeton, New Jersey.

% \aside{I could also talk about language, and developmental trends here,
%   but I don't think I will}
Category membership is also at the core of many more sophisticated theories of induction
\citep[i.e.][]{Osherson1990,Griffiths2009,Kemp2009},
which combine information about category membership
with knowledge about the relationships between various categories.

% Similarity
Other theories of induction are not based on category membership.
Similarity between entities is the most common kind of
non-categorical information proposed to underlie induction:
entities that are similar (i.e. that share many features)
are more likely to also share a new feature
than things that are dissimilar.
This principle forms the basis of many feature-based accounts of induction
\citep{Sloman1993,Rogers2004,Sloutsky2004,Fisher2015},
discussed in Chapter 1.
\citet{Fisher2015} draw a distinction between two kinds of similarity,
based on either overlapping perceptual cues,
or on shared features in our mental representations.
With regard to adults' inferences,
the focus has been on the latter, \emph{representational} similarity.
While it is usually assumed that adults
are not swayed by inappropriate perceptual cues,
overlapping features in our mental representations of entities
form the basis of both \citegap{Sloman1993}{'s} Feature Based Induction model,
and \citegap{Rogers2004}{'s} Semantic Cognition accounts of induction.
\citet{Sloutsky2008,Sloutsky2004} propose
an analogous model to explain children's inferences,
based on perceptual similarity.
%% Interestingly, in contrast to findings \citep[i.e.][]{Murphy2012} that
%% reasoning under uncertainty is based on the most likely categorisation only,
%% inferences made under speeded conditions \citep{Chen2013,Verde2005,Newell2010}
%% are usually made on the basis of shared properties,
%% rather than category membership.

Given the importance of conceptual knowledge such as category membership in induction,
why then might perceptual similarity play a role?
One reason is that similarity is a useful proxy for shared category membership:
categories are collections of things that share properties, 
including visible features,
and things which look alike tend to belong to the same category.
Furthermore, regardless of category membership,
attributes tend to be correlated in the real world:
things that share properties we know of
are more likely to also share novel properties
\cite[see, e.g.,][]{Kemp2012}.
For both of these reasons, under many circumstances 
similarity --- even perceptual similarity ---
and conceptual knowledge will support the same inferences.
However, there do exist problems for which
these two kinds of information disagree,
most notably when reasoning about entities
that look more like members of a different category
than members of the category to which they belong
(that is, visually \emph{atypical} category members).
The archetypal examples of such atypical entities are whales,
which are mammals, but bear closer resemblance to fish
than to other mammals.


% Development
This chapter, like the rest of this thesis, focuses on adults' reasoning.
However, there has been little prior research on the role of
simple perceptual cues in adults' inductive reasoning,
perhaps because it would appear self-evident that
adults rely on conceptual knowledge,
or at least representational similarity.
There has been extensive debate, however,
as to whether young children make use of conceptual knowledge at all,
or if their inferences are simply driven by perceptual similarity.

There are a number of reasons to believe that children may rely on perceptual-cues,
rather than conceptual knowledge, in induction.
Conceptual knowledge must be acquired through instruction and experience,
and so early inferences, by which infants and children make sense of the world,
must be driven by a simpler, associative process like visual similarity
\citep{Westermann2013,French2004}.
In recent years, it has also become apparent that
simple associative mechanisms can produce
inferences that are surprisingly complex and flexible \citep{Sloutsky2008,Hinton2014}.
Indeed, \emph{deep neural networks} ---
neural networks, using simple associative principles but with many layers ---
are at the forefront of modern machine learning and artificial intelligence research
\citep{Mnih2013,Hinton2006}.

Sloutsky \citep[i.e.][see \citealp{Sloutsky2003,Sloutsky2010} for reviews]{
  Sloutsky2008,Sloutsky2007,Sloutsky2004a}
has therefore argued that young children's reasoning
is often based on such perceptual cues,
with reliance on conceptual knowledge arising later in development.
In contrast, Gelman 
\citep[i.e.][see \citealp{Gelman2011a,Gelman2004a} for reviews]{
  Gelman2013c,Rhodes2009,Gelman2007a,Gelman1986}
argues that even toddlers
 % \aside{Or younger?}
rely on conceptual knowledge about the world,
and naive theories \citep{Gopnik2003,Carey2009} 
to make inferences about the world around them.

\citegap{Gelman1986}{'s} triad task,
discussed in Chapter 1,
has been widely used as a testing ground
for these competing accounts.
Recall that in this task,
children were presented with images of two entities
(a flamingo and a bat, for instance),
given their labels (``bird'' and ``bat''),
and told a property of each
(``This bird's legs get cold at night'';
``This bat's legs stay warm at night'').
They were also shown a third species
(a blackbird in this example, labelled ``bird'',
but which more closely resembled the bat)
and asked to indicate which property it would have
(``Do this bird's legs get cold at night, like this bird's,
or stay warm at night, like this bat's?'').
\citet{Gelman1986} report that children as young as four
predominantly resist the perceptual cue,
and reason based on shared category membership --
projecting a property from the flamingo to the blackbird,
in this case.
On the other hand, \citet{Sloutsky2007} present results with artificial categories
that seem to show children ignoring category membership
in favour of perceptual similarity.
\citet{Gelman2013c}, however, demonstrate that
this result only holds under certain circumstances,
specifically, when the nature of the learned categories is obscure
and not obviously relevant to the properties being projected
(i.e. a creature's ratio of buttons to fingers).

% Hybrid theory
In Chapter 1, I introduced \citegap{Bright2014a}{'s} hybrid theory of induction.
This proposes that when reasoning inductively,
we can draw on either simple, associative knowledge --- such as similarity ---
or structured knowledge about the relationships between
entities and the categories they belong to.
The distinction in question in the current chapter,
between perceptual cues and simple conceptual knowledge,
is perhaps a more fundamental one than that made by
\citet{Bright2014a} between associative and structured knowledge.
Nevertheless, their hybrid account could easily
be extended to include it.

A prediction that emerges from this extended version of the hybrid theory
is that even adults' inferences may be influenced by perceptual similarity.
To date, there has been little evidence
suggesting that adults' inferences are driven by perceptual cues.
However, as discussed in Chapter 1,
this may in part be because methods used in previous work
were poorly suited to revealing such conflict,
and so in this chapter I explore this question
using the mouse tracking paradigm.
However, previous studies using this triad task
\citep[i.e.][]{Gelman1986,Sloutsky2007,Gelman2013c}
both with adult and child participants,
have only included trials in which perceptual similarity
and conceptual knowledge have conflicted.
Typically, adults overwhelmingly respond on the basis
of conceptual knowledge on such trials,
while the debate has focused on whether children's responses
are driven by conceptual knowledge \citep{Gelman1986,Gelman2013c}
or perceptual similarity \citep{Sloutsky2007}.
By not including trials in which both cues agree, however,
these experiments do not make it possible to see
if adults' reasoning was influenced by perceptual cues
in addition to conceptual knowledge:
when perceptual cues disagree with conceptual knowledge,
participants may be less likely to give the conceptually-cued response,
or slower to do so, or be otherwise conflicted,
than when the cues agree.

Therefore, in this chapter, I present two experiments, with adult participants,
in which perceptual cues either conflict or agree with conceptual knowledge.
If adult induction is in part driven by perceptual cues,
participants should be less likely to generalise
from a base entity to a conceptually-related one
when the alternative, foil entity is visually similar to the base.
Recording participants' mouse cursor trajectories,
it is also possible to investigate how these kinds of information interact over time.
If participants are driven by \emph{either} perceptual similarity
\emph{or} conceptual knowledge on each trial,
they should move the mouse directly to one or other response.
Alternatively, if participants are initially driven by
quickly-available perceptual cues,
but draw on conceptual knowledge later in the reasoning process,
we may see reversal trajectories, as participants
initially move towards the perceptually-cued option,
and change direction mid-trial.



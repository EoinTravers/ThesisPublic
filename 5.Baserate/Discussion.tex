
\section{Discussion}

In this experiment, I combined mouse tracking with
a forced choice version of the base rate neglect task,
where I manipulated both the descriptions,
and the base rates participants reasoned about.
In doing so, I build on two lines of research.
Much earlier work in the heuristics and biases and dual process literature 
\citep[i.e.][]{Kahneman2011,Kahneman1973}
claims that descriptions are processed easily and automatically by Type 1 processes,
while processing base rate information
requires the optional later engagement
of more effortful Type 2 processes.
As a result, base rates are more likely to be ignored or underweighted.
More recent accounts \citep[i.e.][]{DeNeys2012,Barbey2007}
mostly agree with this fundamental dual process interpretation,
but argue that base rates play a more involved role in reasoning,
either because they can be processed by Type 1 processes 
\citep{DeNeys2012}, 
or because Type 1 and 2 processes operate in parallel
\citep{Sloman1996,Barbey2007}.


Analysis of problems in this experiment where
base rates and descriptions conflicted
provided support for a generic dual process account.
Participants gave the response cued by the description
on 80\% of such trials,
and when doing so were faster to read the description,
faster to respond,
and less likely to initially move the mouse cursor
towards the opposite option than on trials
where they gave the base rate-cued response.
Participants were also much more likely to
initially move the mouse cursor towards the description-cued option
than the base rate-cued option when the cues disagreed (doing so 66\% of the time),
and more likely to change direction if their initial movement
was towards the base rate-cued option.

Most dual processes accounts ---
the exception being a selective model \citep{Klaczynski2000,Chaiken1987}
where one or other kind of processing is engaged for a given task ---
would predict that participants who do give the response cued by the base rate
should be conflicted while doing so,
as they must inhibit the pull of the description-cued response.
This prediction was tested by analysing
all trials where participants could rely on the base rate,
and looking at effect of manipulating the description.

Consistent with this prediction,
the contents of a problem's description had a strong effect
on participants' base rate-cued reasoning,
as participants were most likely to give the base rate-cued response
when this response was also supported by the description (94\% of trials),
less likely when the description was uninformative (69\%),
and as mentioned above, only selected the base rate-cued response
when it conflicted with the description on 20\% of trials.

Similarly, when they did give this base rate-cued response,
participants both spent longer reading the description,
and spent longer responding,
when the description conflicted with the description than when it agreed.
This suggests that
participants had to inhibit the description-cued response
before they could give the base rate-cued response instead.
Unlike previous experiments in this thesis, however,
participants' were not significantly more likely to trace
reversal trajectories towards the base rate-cued option
under conflict, although the trend was in this direction.

The transition probability analysis (Table~\ref{tab:exp5_br_transitions})
similarly showed that descriptions had a pervasive effect
on participants' base rate-driven reasoning,
influencing both their initial movements,
and their ultimate responses regardless of initial movement direction.
The transition probabilities also allow us to
make sense of the non-significant effect of descriptions
on the probability of participants following a reversal trajectory
when giving the base rate-cued response.
Regardless of the base rate,
participants initially moved towards the description-cued option,
where available, on two thirds of trials.
Given that they were no more likely to initially
move towards the alternative option
when the base rate cued it than when it did not,
we must assume that base rates play no role at this point.
Later in reasoning, the influence of the description is even greater, 
dictating 80\% of final responses when it conflicts with the base rate,
and 94\% both when it agreed and when the base rate was uninformative.
At this point, base rates also clearly play a role,
as the response not cued by the description
is given on only 6\% of trials when it is not cued by the base rate either,
but 20\% when it is cued by the base rate.
To simplify, it appears that only the descriptions,
and random variations, have an effect on initial movements,
while later movements are even more influenced by descriptions,
and also slightly influenced by base rates.
The time course data are also in line with this interpretation,
as base rates had no influence on early movements.
When they did show an effect, after 740 msec,
this only slightly reduced the pull of the description-cued option
rather than reversing it,
as was the case in Experiments 1 and 2, for instance
(see the bottom panel of Figure~\ref{fig:exp5_timecourse}).





While these results are consistent with most
dual process accounts of this task,
some accounts, such as the intuitive logic theory \citep{DeNeys2012,Handley2015},
or a parallel-competitive dual process theory \citep{Barbey2007,Sloman1996},
further predict conflict in the opposite direction,
with participants who base their responses on descriptions
nevertheless showing sensitivity to the base rate information,
and experiencing conflict when base rates and descriptions disagree.

Analysis of participants' responses
on trials where the description cued a response showed
some evidence of sensitivity to base rates,
with participants less likely to give the description-cued response
when it conflicted with the base rate
than when the base rate agreed with the description, or was uninformative.
However, on such conflict trials participants
still overwhelmingly gave the response cued by the description.
Analysis of individual differences here
(Figure~\ref{fig:exp5_description_acc})
showed that the majority of participants' responses
appeared to be slightly influenced by the base rate,
with a small minority strongly influenced by it.

In contrast to previous tests of the intuitive logic theory, however,
\citep{DeNeys2008,DeNeys2008a,Franssens2009,Pennycook2012a}
there was no significant evidence that participants were
influenced by base rates on trials where their responses
were driven by the description,
as their reading times, movement initiation times,
response times, and cursor trajectories
did not differ significantly when the base rate
agreed with the description, was uninformative, or disagreed with it.
However, the trends were in the direction predicted by the intuitive logic account,
with participants slower to respond when the base rate conflicted with the description.
An exploratory Bayesian analysis suggested that
the most plausible effect size was a 2.8\% increase in response times under conflict,
although the data was largely inconclusive.
there was considerable uncertainty around this tiny effect.

Analyses of the transition probabilities
and the time course data told the same story.
Participants' initial mouse movements,
and the position of the cursor before 740 msec,
were not influenced by changes in the base rate
on trials where participants could rely on the description.
Instead, participants were slightly less likely
to select the description-cued response option,
or to be on its side of the screen from 740 msec onwards,
when the base rate cued the alternative response.
Taken together, these results suggest that
base rates either dictated participants' response to a problem,
or were almost totally ignored


To recapitulate, the results of this experiment
provide further support for a dual process interpretation
of base rate neglect \citep{Kahneman2002,Kahneman2005},
where, fast, effortless, automatic Type 1 processes
underlie description-based reasoning,
and slower, effortful Type 2 processes underlie
base rate-based reasoning.
Results consistent with this interpretation
were found in all of the analyses reported.
Participants predominantly gave the description-cued response
when the base rate also cued it and when the base rate was uninformative,
and only did so slightly less when the base rate conflicted with the base rate.
Even when giving base rate-cued responses,
participants were conflicted when the description disagreed with the base rate.
Analysis of the cursor trajectories and the time course data
showed that descriptions dictated both early movements (from half a second)
and participants' actual responses.

The data were less consistent, however, with some intuitive logic accounts
\citep[i.e.][]{DeNeys2012,DeNeys2014a,Handley2015}.
These would predict that
even when participants give the response cued by the description,
they should be sensitive to manipulations of the base rate,
and previous studies have found such effects
\citep[i.e.][]{DeNeys2008,Pennycook2012a,Pennycook2014}.
Here, I found no significant effects of manipulating the base rate
on participants' cursor movements or response times
when they gave the description-cued response.
An exploratory Bayesian analysis, however,
showed that participants were very slightly slower (around 30 msec)
on these trials when the base rate disagreed with the description.
Thus, it is not possible to draw strong conclusions
either for or against the intuitive logic account
from the current data.

The absence of a significant intuitive logic effect
on response times in particular was surprising,
as response times have been used as a measure of this conflict
in a number of previous studies \citep[i.e.][]{DeNeys2008,Pennycook2012a},
outlined in Table~\ref{tab:previous_baserate_studies}.
Although it is difficult to say with certainty
why this experiment differs from previous work,
I can identify a number of possibilities.

First, a number of changes were made to 
the procedure used by \citet{DeNeys2008a}
in order for it to be compatible with the mouse tracking paradigm.
As discussed above, the way in which information
was presented here meant that participants were able to
process each trials' information at two points:
before clicking on the ``NEXT'' button to reveal the question (reading time),
and after seeing the question but before responding (response time).
However, even when I combined these times (not reported),
there was little evidence of participants being slower
to give the description-cued response when it conflicted with the base rate.

The mouse tracking paradigm also requires that
participants respond under time pressure,
in this case within 6 seconds.
A visual timer in the centre of the screen
was used in this experiment to reinforce this idea,
filling up over the course of the allowed time.
As this time limit was only exceeded on 2 trials out of 2,000,
and average response times were below 2 seconds in all conditions,
I can be quite confident that participants did reason quickly,
although again this does not include
time spent reading the description.

These response are considerably faster than
those reported in the majority of previous conflict detection studies
that used the base rate paradigm
and measured response latencies, outlined in Table~\ref{tab:previous_baserate_studies}.
Moreover, of these studies, only two required participants to respond before a deadline.
\citet{DeNeys2008} presented participants in an fMRI scanner
with problems almost identical to those used here,
and required them to respond within 8.5 seconds of seeing the questions.
They found that participants were slower to give description-cued responses
when the cues conflicted (3.5 seconds) than any other condition ($\sim$2.8 seconds).
However, it should be noted that these response times were 
almost a second longer than those reported here,
at a time scale where this constitutes a $\sim$66\% increase.
Their result, while moderate in size ($\eta^2_p = .28$),
was also not robustly statistically significant,
with $F(1, 12) = 4.6, p_{rep} = .87$
(corresponding to approximately $p = .05$).

The other study reporting conflict effects at such a short time scale
was Experiment 2 of \citet{Pennycook2014},
where participants were asked to respond within 5 seconds.
Note, however, that this was an unusual base rate experiment.
Participants were instructed before each trial
to base their response on either ``belief'', or ``statistics'',
and in Experiment 2 asked to respond within 5 seconds.
Analysis of response latencies revealed conflict effects in both directions ---
when reasoning based on statistics, participants were slower 
if the description cued the opposite response,
and likewise slower when reasoning based on belief
if the base rate cued the opposite response.
However, these times remain a second or more slower
than those reported in the current experiment.
Furthermore, the magnitude of this effect was extremely small ---
going from 3.70 to 3.79 for statistics-based decisions ---
a small effect size ($\eta^2_p$) of .08.
It should also be noted that in Experiment 2 of \citet{Pennycook2014},
instructions to rely on belief or statistics were manipulated within-participants,
and so the effect within this short time window
could be in part due to task-switching effects \citep{Monsell2003}.
Experiment 3 of the same paper demonstrated
a similar effect with a between-participants manipulation,
but participants were not asked to respond quickly,
and the dependent variables were participants'
probability judgements and confidence ratings,
not their response times.
Therefore, it seems possible that in the current experiment
a) participants responded too quickly in most cases to detect 
the conflict between their responses and the base rate;
b) having not been explicitly told to use the base rate on 50\% of trials,
participants may have been less sensitive to the base rate
than those in  \citet{Pennycook2014}.

% \input{previous_baserate_studies_table}

\begin{table}
  \centering
  \caption[Prevous base rate neglect studies.]{
    Number of base rate-cued responses under conflict,
    and response latencies for description-cued responses
    when cues either agreed or conflicted,
    in previous base rate studies.}
  \label{tab:previous_baserate_studies}
  \rotatebox{90}{
    %% \begin{tabular}{ L{4cm} L{4cm} L{2cm} L{2cm} L{2cm}}
    \begin{tabular}{ p{.4\textwidth} p{.4\textwidth} p{.2\textwidth} p{.15\textwidth} p{.15\textwidth}}
      %% \begin{tabular}{\textwidth}{p{5cm}p{4cm}LLL}
      \toprule
      Study                                                                               &  Procedure                                          &  BR responses                &  No-conflict RT &  Conflict RT   \\[.75cm]
      \midrule                                         
      \citet{DeNeys2008a}                                                                 &  fMRI                                               &  45\%                        &  2.8            &  3.75          \\[.75cm]
      \citet{DeNeys2008}                                                                  &  Standard                                           &  22\%                        &  $\sim$14       &  $\sim$18      \\[.75cm]
      \citet{Franssens2009}                                                               &  Secondary load                                     &  47\%~(no~load) 35\%~(load ) &  $\sim$14       &  $\sim$17      \\[.75cm]
      \citeauthor{DeNeys2009a} (\citeyear{DeNeys2009a};~Exps.~2--4)                       &  Standard                                           &  $\sim$33\%                  &  $\sim$14       &  $\sim$17      \\[.75cm]
      %% \citet{DeNeys2011b}                                                              &  Confidence ratings                                 &  20\%                        &  NA             &  NA            \\[.75cm]
      \citeauthor{Pennycook2012a} (\citeyear{Pennycook2012a};~Exp.~4; extreme~base~rates) &  Standard                                           &  24--26\%                    &  16             &  20            \\[.75cm]
      \citeauthor{Pennycook2012b} (\citeyear{Pennycook2012b};~Initial responses)          &  Two-response paradigm \mbox{(probability~ratings)} &  N/A                         &  12.8           &  13.4          \\[.75cm]
      \citeauthor{Pennycook2014} (\citeyear{Pennycook2014};~Exp.~1)                       &  Belief/statistics instructions                     &  N/A                         &  $\sim$13       &  $\sim$14      \\[.75cm]
      \citeauthor{Pennycook2014} (\citeyear{Pennycook2014};~Exp.~2)                       &  Belief/statistics instructions, 5~second~deadline  &  N/A                         &  $\sim$3.8      &  $\sim$3.7     \\[.75cm]
      \bottomrule
    \end{tabular}
  }
  %% \end{tabulary}
\end{table}


Another factor may be the scarcity of base rate-cued responses
in general in the current experiment.
Reasonably, one would expect conflict detection to be 
related to participants' responses:
experiments that yield many base rate-cued responses
should also yield greater detection of conflict
on problems in which the description-cued response is given.
In this experiment, on the other hand,
the base rate-cued response was given on only 20\% of conflict problems,
perhaps unsurprisingly given the fast response times here,
and the robust finding that base rate responses
are slower than description responses.
Consequently, it may not be so surprising that
an experiment which yielded relatively few base rate-cued responses
should also show little influence of base rates 
on more subtle measures such as response time.

Lastly, recent work 
\citep[e.g.][]{DeNeys2010,Mevel2014} has highlighted
that there are individual differences in
this conflict detection process ---
not all participants are slower, or less confident,
when their responses conflict with base rates,
or with logical principles.
At present, we know relatively little about the factors
that make some participants, but not others,
sensitive to these conflicts.
Therefore, given that almost all studies in this area
reveal some participants for whom
the conflict detection effects do not hold,
it should perhaps not be surprising that 
these effects are not found in every study.

\subsubsection{Conclusions}

To conclude, this chapter reports one of the first uses of
the mouse tracking paradigm to investigate the interaction of
dual processes during reasoning.
Results are consistent with a default-interventionist
dual process accounts \citep{Evans2006,Kahneman2002,Evans2013a},
by which descriptions are processed easily and automatically
by Type 1 processes,
but base rates, thought to require Type 2 processes to process.
play less of a role in reasoning on most trials.
In fact, when the base rates were attended to,
they tended to dictate participants' responses outright.
The rich, temporal dynamics of the data collected
using this paradigm reveal much more about the underlying processes,
for instance that only descriptions influence
the initial direction of participants' cursor movements,
and that the descriptions have a discernible effect on cursor movements
from around 500 msec, while the weaker effect of base rates
is not apparent until around 750 msec.

A number of recent studies have also shown evidence
of conflict in the opposite direction,
as base rates have some effect on participants
even when their responses appear to be dictated by descriptions alone.
The current data, however, did not support this idea,
although this may in part be due to the ways in which
the experimental paradigm had to be adapted
to suit the mouse tracking paradigm,
in particular, the extremely faster response times,
and unusual presentation of information.
Despite this, however, the current chapter demonstrates 
another point at which conflict can be found in reasoning,
when the right kind of data is collected.

% % % % % % % % % % % % % % % % %
% % % % % % % % % % % % % % % % %
% % % % % % % % % % % % % % % % %
% % % % % % % % % % % % % % % % %












\section{General Discussion}


In these two experiments
participants completed versions of the inductive triad task
where they were asked to generalise a property from a base category
to one or other response category.
They could rely either on conceptual knowledge,
and so generalise the property to the category that
belonged to the same category as the base,
or on perceptual cues,
and generalise to the category that looked like the base.
In both experiments, I manipulated whether the cues agreed or disagreed,
so that on conflict trials perceptual cues would lead participants
to select the foil response option rather than the
conceptually-related one.
On these conflict trials, participants were
more likely to select the foil category.
Furthermore, they were more likely to make
fast initial mouse movements towards the foil,
which they often overrode to select the correct category instead.
These results are consistent with \citegap{Bright2014a}{'s} hybrid theory of induction.

I also raised the question of how perceptual cues
and conceptual knowledge might interact during induction.
One option was that participants
selectively relied on perceptual cues \emph{or} conceptual knowledge.
The alternative was that participants were
initially driven by perceptual cues,
but sometimes overrode theses cues when they realised them to be inappropriate,
and relied on conceptual knowledge instead.

The results showed that participants,
although usually initially moving towards the perceptually-cued foil,
sometimes moved directly towards the conceptually-cued option,
with these movements generally taking longer to initiate
than those towards the foil.
There are two possible explanations for such trials.
It may be that participants simply draw on conceptual knowledge here,
a process which takes longer, and then act on it.
Alternatively, participants on these trials
may have first processed the perceptual cues,
but inhibited them and replaced them with their conceptual knowledge
before initiating their cursor movement.
It is not clear at present how these possibilities could be disentangled,
and so for the time being this particular issue remains an open question.

In Experiment 2, I manipulated the kinds of properties participants reasoned about:
either specific properties, that were saliently related
to the distinction between the two categories,
or generic properties, that were not.
This manipulation mainly affected what participants did
on trials where they initially moved towards a perceptually-cued foil option.
Participants reasoning about specific properties were
significantly more likely to override their initial movement
and select the correct option instead
than those reasoning about generic properties.
The manipulation did not make participants any less likely
to initially move towards the foil option,
and its influence on the time course data
emerged relatively late in reasoning (see Figure~\ref{fig:exp2_foil_side_timecourse}).
It also had no influence on control trials,
where participants almost invariably selected the correct option.
Therefore, it would appear that this manipulation
mainly served to make participants more likely
to inhibit their perceptually-driven responses
on trials in which they were initially driven towards giving them.
This is consistent with much previous work,
both using the current paradigm \citep{Gelman2013c}
and other inductive tasks \citep{Heit1994,Ross1999},
indicating that the kind of information people draw on in inductive reasoning
is contingent on the nature of the properties to be projected.
A future question raised by this result concerns
whether this manipulation serves to make
participants more likely to inhibit perceptual cues,
or if it makes certain structured knowledge
easier to retrieve by cuing or priming it.

The purpose of these experiments
was to investigate adults' inductive reasoning,
and the results are consistent with \citegap{Bright2014a}{'s} hybrid account.
Specifically, they suggest that induction cannot be explained entirely
by either accounts based on unstructured associative knowledge
such as (perceptual or representational) similarity
\citep[i.e.][]{Sloman1993,Rogers2004,Sloutsky2004,Fisher2015}
or by purely structured, conceptual knowledge
\citep[i.e.][]{Osherson1990,Griffiths2009,Kemp2009,Gelman1986}.
Instead, both kinds of information appear to influence adults' reasoning,
with simple perceptual similarity drawn on earlier in the process,
and perhaps serving as a default.
However, these results also have implications for theories of children's reasoning.
As discussed in Chapter 1, and earlier in the current chapter,
a number of experiments have claimed to show that
young children's inferences are either
driven by conceptual knowledge
\citep{Gelman2013c,Rhodes2009,Gelman2007a,Gelman1986},
or that they are driven by perceptual similarity
\citep{Sloutsky2008,Sloutsky2007,Sloutsky2004a}.
These previous experiments \citep[i.e.][]{Gelman1986,Sloutsky2007,Gelman2013c},
however, have focused on a binary question:
do children draw on perceptual cues, \emph{or} on conceptual knowledge during reasoning?
Therefore, these experiments only presented participants with conflict trials,
and their responses were classed as either
consistent with reliance on perceptual similarity,
consistent with reliance on conceptual knowledge,
or not significantly different from chance in either direction.
As discussed in Chapter 1,
an experimental control condition, of the type used here,
where both cues agree, makes it possible to discover
not only which cue dominates when both conflict,
but also whether the neglected cue,
perceptual similarity in this case,
has any influence at all.

Therefore, these results with adults suggest a new interpretation
of the developmental data:
if both perceptual similarity and conceptual knowledge influence adults' reasoning,
they likely both also play a role in children's inferences.
This perspective may make sense of
apparently contradictory results in the developmental literature,
where children seem to draw on conceptual knowledge
in some scenarios but not others.
It is likely that these studies differ
in terms of the factors which make participants
more or less likely to inhibit initially influential perceptual cues.
Thus, while children may have access to both perceptual cues
and information about conceptual knowledge across all of these experiments,
they are more likely to inhibit the former in favour of the latter
when categories differ at a high, ontological level,
when entities are more easily categorised,
or when the properties under consideration are
conceptually related to the distinction between the categories
\citep{Gelman2013c}.
In short, the current results suggest that
developmental researchers should be less concerned
about \emph{whether} children rely on similarity or on conceptual knowledge,
and instead ask \emph{when} do children rely on either form of information.



% \aside{ This is where Newall 20 questions with nature bit comes in! }







% Having shown that perceptual and conceptual knowledge
% are activated during induction,
% our next question is how to these representations
% interact during the reasoning process.
% \aside{I need to look at trajectories when the wrong response was chosen.}
% Unlike a number of mouse tracking studies
% that found graded, continuous attraction towards both responses \citep[i.e.][]{Spivey2005},
% we found that participants moved directly towards one or other response,
% and on many trials moved initially towards one, before changing direction mid-flight.
% Earlier trajectories were more likely to be directed
% towards the perceptually-cued response option
% while later movements moved towards the conceptually-related option.
% This is consistent with the idea that only one representation
% can exist in working memory at a time.








% the abstract

This thesis explores conflict during reasoning,
using a mouse tracking paradigm that provides a measure of
participants' instantaneous attraction
towards competing response options.
It focuses on two kinds of conflict:
conflict between competing sources of information in inductive reasoning,
and conflict between fast, automatic Type 1 processes
and slower, deliberate Type 2 process in dual process accounts of reasoning.
Going beyond traditional analyses of participants' responses,
the mouse tracking data reveal
under what circumstances conflict occurs,
and at what points in time participants were influenced by different factors,
as well as something of the qualitative nature of this conflict.
Chapter 1 reviews previous work on conflict in reasoning,
and methods used in the past to study conflict in cognition.
Chapter 2 introduces the details of the mouse tracking paradigm used in subsequent chapters.

Chapters 3 and 4 explore conflict between different sources of information.
In Chapter 3, two experiments are presented
that pit perceptual cues (visual similarity)
against conceptual knowledge (shared category membership)
in an inductive reasoning task,
using both real and artificial stimuli.
This reveals that participants are initially driven by perceptual cues,
and only later draw on conceptual knowledge,
and also that they are more likely
to draw on this knowledge when the properties being reasoned about
were related to the distinction between the categories.
In Chapter 4, two further experiments are presented
that pit associative and structured knowledge
against each other in an induction task.
Participants were again influenced by both kinds of information,
but it was less clear to what extent
both forms of knowledge interacted on a single trial.

Chapters 5 and 6 explore conflict in dual process theories of reasoning.
Chapter 5 reports a judgement task
where participants attempted to categorise people,
and could rely on either descriptions of them
or on background base rate statistics.
Consistent with a default-interventionist dual process model,
participants were strongly influenced by descriptions from early in reasoning,
but only drew on statistical information,
and sometimes overrode description-cued responses on the basis of it, later on.
Chapter 6 presents a variation the mouse tracking paradigm,
as participants' cursor movements over up to a minute
were analysed as they chose between response options on
a task where Type 1 processes cue an intuitively appealing incorrect response,
the Cognitive Reflection Test.
Results here are again consistent with a default-interventionist perspective,
as movements were apparently initially driven by Type 1 processes,
and correct responses involving an initial attraction towards the heuristic option.

Chapter 7 discusses the implications of these findings,
both for theories of reasoning and for accounts of conflict in cognition.
It also explores how the methods used here
may prove fruitful in future reasoning research.




%% This PhD is concerned with the causal Bayesian framework account of
%% probabilistic judgement (Krynski & Tenenbaum, 2007). which posits that
%% accurate Bayesian reasoning is contingent upon the reasoner being able
%% to intuitively represent evidence in terms of a sufficient causal
%% model. According to this account, base rate neglect whereby reasoners
%% fail to consider the prior probability of the hypothesis, irrespective
%% of the evidence - can be overcome by explicitly clarifying the causal
%% basis of all of the given evidence.  Chapter 1 reviews the literature
%% on base rate neglect; details the main accounts of base rate neglect
%% in Bayesian reasoning; considers some issues with the causal Bayesian
%% framework; and finally outlines the main aims of the PhD.  Chapter 2
%% presents two dual-task experiments which aimed to test the
%% intuitiveness of causal facilitation. Whilst secondary load was not
%% found to have an overall effect on reasoning, subsequent non-load
%% experiments in the chapter found similar levels of causal facilitation
%% to the dual-task experiments. leading to the overall conclusion that
%% facilitation occurs outside of working memory considerations. Chapter
%% 3 indicated that the causal facilitation effect was present only • in
%% the absence of an additional intervention designed to highlight nested
%% set relations in the data, indicating that reasoners may not employ a
%% strictly causal model, but instead represent different causes as
%% interrelated sets of data.  Chapter 4 demonstrated that the causal
%% facilitation effect was limited to reasoners of sufficiently high
%% numeracy, which explained the consistently lower levels of Bayesian
%% responding reported throughout the rest of the PhD.  • Overall, the
%% thesis furthers our understanding of how mental representations can
%% positively influence judgements over the classical, purely statistical
%% approach. Clarifying the causal basis of the given evidence can help
%% reasoners of good numerical ability to intuitively recognise the set
%% relations between data, leading to Significant improvements in
%% performance.


%% This thesis concerned with the factors underlying both selection
%% and use of evidence in the testing of hypotheses. The work it
%% describes examines the role played in hypothesis evaluation by
%% background knowledge about the probability of events in the
%% environment as well as the influence of more general constraints.
%% Experiments on information choice showed that subjects were sensitive
%% both to explicitly presented probabilistic information and to the
%% likelihood of evidence with regard to background beliefs. It is argued
%% - in contrast with other views in the literature - that subjects'
%% choice of evidence to test hypotheses is rational allowing for certain
%% constraints on subjects' cognitive representations. The majority of
%% experiments in this thesis, however, are focused on the issue of how
%% the information which subjects receive when testing hypotheses affects
%% their beliefs. A major finding is that receipt of early information
%% creates expectations which influence the response to later
%% information. This typically produces a recency effect in which
%% presenting strong evidence after weak evidence affects beliefs more
%% than if the same evidence is presented in the opposite order. These
%% findings run contrary to the view of the belief revision process which
%% is prevalent in the literature in which it is generally assumed that
%% the effects of successive pieces of information are independent. The
%% experiments reported here also provide evidence that processes of
%% selective attention influence evidence interpretation: subjects tend
%% to focus on the most informative part of the evidence and may switch
%% focus from one part of the evidence to another as the task
%% progresses. in some cases, such changes of attention can eliminate the
%% recency effect.  In summary, the present research provides new
%% evidence about the role of background beliefs, expectations and
%% cognitive constraints in the selection and use of information to test
%% hypotheses. Several new findings emerge which require revision to
%% current accounts of information integration in the belief revision
%% literature.


%% Making judgements often involves integrating multiple pieces of information, or cues, in
%% the environment. While experts, such as physicians, are able to make accurate judgements
%% from multiple cues, they often have poor insight into how they make their inferences. This
%% provides some indication that judgement is influenced by knowledge that is implicit and
%% inaccessible to verbal report. In the present thesis, the cognitive processes involved in
%% multiple cue judgement were explored by training participants on a small number of novel
%% cues using the multiple cue probability learning (MCPL) paradigm. In a training phase,
%% participants predicted a criterion and received outcome feedback in response to each
%% judgement. Learning and judgement in these tasks is often assumed to draw on explicit
%% hypothesis-testing processes. However, a great deal of research suggests that implicit as
%% well as explicit processes can contribute to performance on complex tasks. In eight
%% experiments, several methods were used to examine the role of explicit and implicit
%% processes in multiple cue judgement. While concurrent working memory loads failed to
%% disrupt judgements after learning, we nevertheless found clear evidence that explicit
%% processing is involved in the learning of negative, but not positive cues. Performance on
%% such tasks was correlated with individual differences in working memory capacity, as well
%% as measures of explicit knowledge obtained in the learning process. The results are
%% discussed with respect to dual process theories of learning, judgement, and reasoning. The
%% findings of the present thesis indicate that multiple cue judgement is best viewed within a
%% dual process framework.




%% Current theories of category-based inductive reasoning can be
%% distinguished by the emphasis they place on structured and
%% unstructured knowledge. Theories which draw on unstructured knowledge
%% focus on associative strength, or temporal and spatial contiguity
%% between categories. In contrast, accounts which draw on structured
%% knowledge make reference to the underlying theoretical frameworks
%% which relate categories to one another, such as causal or taxonomic
%% relationships. In this thesis, it is argued that this apparent
%% dichotomy can be resolved if one ascribes different processing
%% characteristics to these two types of knowledge.  That is,
%% unstructured knowledge influences inductive reasoning effortlessly and
%% relatively automatically, whereas the use of structured knowledge
%% requires effort and the availability of cognitive
%% resources. Understanding these diverging processes illuminates how
%% background knowledge is selected during the inference process.  The
%% thesis demonstrates that structured and unstructured knowledge are
%% dissociable and influence reasoning in line with their unique
%% processing characteristics. Using secondary task and speeded response
%% paradigms, it shows that unstructured knowledge is most influential
%% when people are cognitively burdened or forced to respond fast,
%% whereas they can draw on more elaborate structured knowledge if they
%% are not cognitively compromised. This especially evident for the
%% causal asymmetry effect, in which people make stronger inferences from
%% cause to effect categories, than vice versa. This Bayesian normative
%% effect disappears when people have to contend with a secondary task or
%% respond under time pressure.
%% The next experiments demonstrate that this dissociation between
%% structured and unstructured knowledge is also evident for a more
%% naturalistic inductive reasoning paradigm in which people generate
%% their own inferences.  In the final experiments, it is shown how the
%% selection of appropriate knowledge ties in with more domain-general
%% processes, and especially inhibitory control. When responses based on
%% structured and unstructured knowledge conflict, people‟s ability to
%% reason based on appropriate structured knowledge depends upon having
%% relevant background knowledge and on their ability to inhibit the lure
%% from inappropriate unstructured knowledge.  The thesis concludes with
%% a discussion of how the concepts of structured and unstructured
%% knowledge illuminate the processes underlying knowledge selection for
%% category-based inductive reasoning. It also looks at the implications
%% the findings have for different theories of category-based induction,
%% and for our understanding of human reasoning processes more generally.

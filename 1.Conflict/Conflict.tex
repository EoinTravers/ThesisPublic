%%% Local Variables: ***
%%% mode:latex ***
%%% TeX-master: "Chapter.tex"  ***
%%% End: ***

\section{Conflict in Cognition}
\label{sec:chapter1-conflict}

At this stage, I have covered
the specific cases  of conflict in reasoning
that I focus on in this thesis.
Reasoning, however, is not the only cognitive domain
in which conflict occurs,
and the majority of previous research on conflict in cognition
focuses on simpler cognitive tasks.
Of course, high-level cognition
is built upon the operation of these lower level processes \citep{Barsalou2007},
and so understanding conflict in these simpler processes
may advance our understanding of conflict in reasoning.

%% Within a process
%% This thesis will not delve into conflict at the biological level.
%% Therefore, the simplest form of conflict I will discuss
%% is that which occurs within a single low-level cognitive process.
%% It has been proposed argued extensively that
%% some kinds of cognitive processes are \emph{associative},
%% that is, they rely on unstructured statistical relationships such as
%% similarity and temporal or spatial contiguity,
%% rather than explicit rules \citep{Sloman1996}.
%% Such processes have variously been referred to as
%% cognitive modules \citep{Fodor1983},
%% implicit processes \citep{Greenwald1995,Evans1996},
%% autonomous processes \citep{Stanovich2009},
%% and automatic processes \citep{Schneider1977}, % Talking about learned processes
%% amongst other terms.

\subsection{A Mind Divided}
\label{subsec:chapter1-conflict-divided}

The simplest form of conflict I will consider
occurs \emph{within} individual low-level cognitive processes.%
\footnote{
  Conflict occurs in the brain at yet lower levels than this,
  for instance when a single neuron receives
  both excitatory and inhibitory signals,
  and must \emph{decide} whether or not to fire.
  I limit this discussion to cognitive conflict for brevity.
  }
Many of these processes perform what can be broadly defined as categorisation,
for instance categorising aural stimuli as words
\citep{McClelland1986,Norris1994}.
Such processes are commonly
modelled using artificial neural networks,
otherwise known as connectionist models
\citep{McClelland1986, Rogers2004, Kruschke1992}.
These represent mental states using patterns of activation
across a network of connected neuron-like nodes.
The connections between nodes,
or the influence the activation of each node has on its neighbours,
provide a form of long-term information storage.
The behaviour of a network is therefore dictated
both by the long-term patterns stored in its connection weights,
and by the external activation of its \emph{input nodes}.
These artificial networks, and the real ones in the brain,
can be thought of as mapping
specific patterns of input to their corresponding outputs,
such as mapping phonological or visual patterns to words in the mental lexicon.
Presented with ambiguous inputs, such as a sound intermediate between two common words,
such networks may be forced to produce either
the output that most closely corresponds to that input,
or an output that compromises between the two learned patterns.
Given that conflict within such processes
can often be resolved with a compromise,
I believe the term \emph{competition}
accurately describes this form of conflict.

More advanced \emph{recurrent} networks
\citep{Elman1990}               % Also Jordan, Boltzman, etc.
have patterns of activation that develop over time,
with activation at time $t$ influenced by both the network's inputs,
and its previous activation, at time $t - 1$.
Crucially, the state of such networks can develop over time,
so that when presented with input that partially supports multiple outputs
-- a sound intermediate between two words, such as ``pin'' and ``bin'', for instance --
the network will initially produce an output pattern
intermediate between the two,
before settling over time into a \emph{steady state}
consistent with one or other word,
by a process of \emph{recurrent competition} \citep{Spivey2007}.
This competition is thought to resolve itself internally,
without the need for top-down control,
although it is known that its current state
is available to other cognitive process and to the motor system
\citep[i.e.][]{Spivey2005}.
The dynamics of conflict and competition in such processes
have been extensively studied from a number of perspectives,
including as sequential sampling models, % Cites!
and non-linear dynamical systems
\citep{Spivey2006, Spivey2007,Beer2000,Port1995,Freeman2011d}.
However, few theories of reasoning are restricted
to the operation of a single such process \citep[but see][]{Rogers2004},
and so this particular kind of conflict will not be the focus of this thesis.
I will, however, revisit this idea in Chapter 7.

%% Between processes
Conflict can also occur between these low-level processes,
The paradigmatic example of this conflict is the Stroop task \citep{Stroop1935}.
In this, participants are shown colour names,
printed in colours that may agree, or disagree with the printed word,
and asked to state the colour of the ink.
Participants struggle on trials in which the ink colour and printed word differ,
making more mistakes and responding more slowly.
As the process of reading the printed work is
stronger that of identifying its colour ---
it provides the \emph{prepotent response} ---
top-down control processes (discussed below)
are required to successfully complete this task,
specifically, the prepotent response must be \emph{inhibited}
in order for the weaker response to be produced.
% De Neys thinks that reasoning looks like this.

In some cases, this conflict may be resolved as a compromise,
similar to competition within a single low-level processes.
In the McGurk effect \citep{Mcgurk1976},
participants watching a video of a person pronouncing the syllable {[}ga{]},
while hearing the syllable {[}ba{]},
report perceiving a syllable phonologically intermediate between the two, {[}da{]},
indicating that the outputs of both processes 
were combined to form a conscious perception.
In other situations, however, such compromise is not possible.

So far, I have described these conflicting processes
from a bottom-up perspective:
multiple processes, or cues within a single process,
are set in conflict with each other,
and therefore compete until one or other outcome occurs,
or a compromise is reached.
From this point of view, it would appear that we as conscious agents
have much control over our mental lives,
but rather we are at the mercy of these
conflicting autonomous processes.
In the past, researchers in a number of traditions
have argued that this is in fact the case,
and that our subjective feelings of conscious control
are illusionary, or irrelevant
\citep[e.g.][]{Skinner1965,Gibson1986}
In mainstream psychology and neuroscience, however,
it is usually accepted that \emph{we} are capable of
monitoring, directing, controlling, and inhibiting
these low-level processes in the service of our goals,
a set of abilities known as \emph{cognitive control}.
Cognitive control plays an important role in theories of conflict in cognition,
and has recently begun to be addressed in theories of reasoning as well \citep{DeNeys2012}.
In the next section, I will outline some aspects of
the literature on control relevant to this thesis.


\subsection{Cognitive Control}
\label{subsec:chapter1-conflict-control}

The idea that some cognitive processes are involved
in the monitoring and regulation of others is a pervasive one
throughout both psychology,
where they are commonly known as executive functions, or the central executive,
and neuroscience,
where they are also referred to as cognitive control
\citep[see, e.g.,][]{Diamond2013,Botvinick2014,
  Botvinick2001,Munakata2011}.
I use term ``cognitive control'' here,
as it better captures the full range of processes under consideration.
Core control functions include
directing cognitive processes to achieve higher-order goals,
monitoring the performance and accuracy of other processes,
%% (including high-level \emph{metacognition}),
detecting when processes come into conflict,
and inhibiting and controlling
automatically activated cognitive processes and representations
and automatically executed actions.
While all of these functions undoubtedly play a role in reasoning,
of particular relevance in this thesis
are the detection of conflict, and inhibitory control.

First, the detection of conflict.
Above, I discussed the dual process conflict monitoring account
put forwards by De Neys \citep[e.g.][]{DeNeys2008a}.
This is an extension of the more general \emph{conflict monitoring theory}
\citep{Botvinick2014,Shenhav2013,Botvinick2004a,Botvinick2001,Yeung2004}.
According to this account,
the Anterior Cingulate Cortex (ACC),
a deep cortical region adjacent to the frontal and parietal lobes,
receives signals from neural regions
performing many other cognitive processes,
and automatically detects when these signals conflict.
Consistent with this theory, the ACC is known to activate
a) during \emph{response inhibition} tasks such as the Stroop task,
where participants must prevent themselves from
responding by reading the word
%% where irrelevant information
%% that would otherwise drive responding must be ignored;
b) for \emph{under-determined responding},
where the available information supports multiple possible responses,
such as in the stem completion task \citep{Palmer2001}; and
c) for errors on speeded tasks
where participants accidentally give
the wrong response despite knowing the correct one.
When conflict is detected,
the ACC allocates more attentional/inhibitory resources
to the processes in question,
in an attempt to resolve the conflict in a way
consistent with higher-order goals.
In line with this, ACC activation on conflict trials
predicts diminished conflict on the following trial \citep{Rabbitt1966}:
participants experience conflict,
and so allocate more control resources to the task,
and therefore are better able to resolve this conflict on the next trial.

Crucially, this conflict detection processes is modelled
using an artificial neural network \citep{Yeung2004},
whereby conflict is detected automatically
as a consequence of the meeting of various neural pathways in the ACC,
rather than through explicit conscious monitoring.
\citet{Shea2014c} contrast this implicit, automatic form of cognitive monitoring
with more effortful, explicit metacognition
in their dual system account of metacognitive control.
Their \emph{Type 1 metacognition} is
the implicit, automatic representation of
the states of cognitive processes by other processes,
such as the conflict detection of the ACC,
or a process which monitors the variability
or uncertainty of another \citep[see, e.g.,][]{Drugowitsch2012}.
Their \emph{Type 2 metacognition} is the conscious,
effortful reflection on our own mental states
for which the term metacognition is traditionally reserved.
In effect, they argue that
some cognitive control processes are examples of implicit metacognition,
and that traditional metacognition can be considered
an explicit form of cognitive control.
They further suggest that Type 2 metacognition evolved
as a means for people to communicate their metacognitive states,
and to facilitate group decision making \citep[see also][]{Mercier2011}.

The relevance of the conflict monitoring theory
to reasoning research should be clear at this point,
and indeed many aspects of it are incorporated into
more recent intuitive logic accounts \citep[e.g.][]{DeNeys2012,Pennycook2015};
generally, people can easily detect
when multiple responses come into conflict during cognition.
The dual systems account of metacognition proposed by \citet{Shea2014c},
however, has received less attention in the reasoning literature,
but maps clearly onto existing dual process accounts.
In particular, while classical default-interventionist theories
would attribute the detection of biases in heuristic reasoning
to explicit Type 2 metacognition,
it is the contention of the intuitive logic theory
that this detection is instead achieved by Type 1 metacognition.

Of course, once conflict has been detected,
additional control processes are required
to inhibit unwanted processes, mental representations, and actions.
This requires the inhibitory control processes \citep{Diamond2013},
specifically \emph{cognitive inhibition}
--- the inhibition of cognitive processes or mental representations --
and response inhibitions
--- the inhibition of inappropriate actions themselves.
These inhibitory processes have long been associated with
the prefrontal cortex (PFC), adjacent to the ACC.
In an influential account, \citet{Miller2001} propose that
inhibitory control is achieved by the PFC
by sending biasing signals to regions implementing other processes,
which affect the ongoing competition in those regions
to reach the conclusions consistent with higher-level goals.
Therefore, a loop is formed,
from cortical regions implementing basic cognitive processes
that come into conflict,
to the ACC where conflict is detected,
to the PFC, which in turn influences the conflicting cortical processes
\citep{Yeung2004}.
While this thesis does not test theories of executive control,
it seems apparent that they must play a role
in conflict during reasoning.
I will return to these processes in Chapter 7.


%% The executive functions (EFs) are cogntive processes
%% that monitor and control other processes \citep{Diamond2013} \aside{Possibly need more references here}
%% and integrate them as part of high-level cognition.
%% For our purposes, two executive functions in particular are relevant:
%% inhibition, which blocks or overrides other processes,
%% and monitoring processes, which represent the state of other processes,
%% and identify when inhibition or other EFs are required.

%% % Inhibition
%% Inhibitory control consists of processes that
%% ``{[}control{]} one's attention, behaviour, thoughts, and/or emotions
%% to override a strong internal predisposition or external lure'' \citep{Diamond2013},
%% and can be further divided into response inhibition,
%% (the ability to hold back prepotent responses),
%% cognitive inhibition
%% (preventing unwanted mental representations from entering Working Memory),
%% and interference control (the allocation of attention).
%% In our framework, response inhibition is required
%% when conflict arises between
%% competing associative processes,
%% or between associative process and Working Memory.
%% Cognitive inhibition, on the other hand, 
%% serves as a gatekeeper for Working Memory,
%% and so plays a focal role in reasoning
%% when multiple representations are available.
%% \aside{There is more I can include here, including neurological stuff,
%%   but it's probably not necessary.
%%   I also had a section about cognitive flexibility here, in
%%   the previous draft, although it's might not be relevant. }

%% % Conflict monitoring
%% Although inhibitory control provides us a means of 
%% resolving conflict between cognitive processes,
%% we also need some way of knowing which processes
%% need to be inhibited, and when.
%% In classical theories of Executive Control \citep[e.g.][]{Baddeley1974},
%% this role was filled by a \emph{central executive},
%% a system that directed and monitored the other ``slave'' systems.
%% This mechanism has been criticised, however, as a homunculus:
%% a mind-within-the-mind that supposedly explains the phenomena,
%% but hasn't itself been explained \citep{Parkin1998}.
%% The Conflict Monitoring theory \citep{Botvinick2004a, Botvinick2014, Yeung2004}
%% is an attempt to eliminate this homunculus from cognitive control.
%% Unlike the central executive,
%% the actual mechanisms underlying the conflict monitoring account
%% can be specified in connectionist networks \citep[see][]{Yeung2004},
%% in which conflict detection occurs as 
%% an emergent consequence of the layout of the network,
%% rather than relying on an anthropomorphic controller.
%% It proposes that the dorsal Anterior Cingulate Cortex (ACC)
%% serves as a general purpose conflict monitoring system,
%% responding to the cuing of conflicting responses by competing cognitive processes.

%% Consistent with this theory, ACC activation correlates with known sources of conflict,
%% in a) \emph{response inhibition} tasks such as the Stroop task,
%% in which irrelevant information that would otherwise drive responding must be ignored;
%% b) \emph{under-determined responding},
%% where the available information supports multiple possible responses,
%% such as the stem completion task \citep{Palmer2001}; and
%% c) errors on speeded tasks
%% in which participants are aware of the correct response,
%% but fail to produce it.
%% According to the conflict monitoring theory,
%% the role of the ACC is to activate the necessary executive control processes,
%% localised to adjoining prefrontal regions, in response to conflict.
%% \aside{Need to include references for these findings
%%   (can be found in \citet{Botvinick2014,Botvinick2004})}
%% A notable finding in support of this idea is that
%% experimental trials that elicit conflict,
%% either because participants responded incorrectly,
%% or because of incongruency in the stimuli presented,
%% lead to greater activation of control processes,
%% measured both behaviourally, and by means of EEG/fMRI,
%% and additionally that this activation of control processes
%% is correlated with the strength of ACC activation on the previous trial.
%% %% \aside{Cite a few}
%% % I should cite more sources here. I will see what's cited in
%% % Botvinick et al (2004).  Also, could talk about De Neys' work here,
%% % but I think it's more approparite later, after dual process theories
%% % have been introduced.

%% \aside{Put Shea's work here?}

%% %% Why reasoning is different
%% However, there is something that sets the study of conflict in
%% high-level reasoning and decision making
%% apart from that of conflict in the lower-level processes.
%% In high-level cognition, we draw upon a set of flexible,
%% domain general cognitive processes,
%% commonly known as the Working Memory system \citep[WM;][]{Baddeley1974, Baddeley1992}.
%% WM is capable of storing, manipulating, and updating
%% both perceptual and abstract information,
%% and is thought to be essential to flexible, goal-directed
%% high-level processes such as reasoning.

%% WM, thought to be strongly localised within the prefrontal cortex (PFC),
%% and the dorsolateral prefrontal cortex (dPFC) in particular \citep{Kane2002},
%% is capable of storing, manipulating, and updating both perceptual and abstract information,
%% is usually held to underlie uniquely human skills such as abstract, high-level thinking.
%% Importantly, while WM is capable of more flexible, abstract operations
%% than the associative processes mentioned above,
%% unlike them its capacity is extremely limited:
%% only a small number of \emph{objects} can be stored in WM at any one time
%% \citep{Miller1956}. \aside{Cite more recent sources?}
%% Similarly, WM is thought to be capable of representing only
%% a single representation of the world at any time \citep{Evans2006},
%% or, in other words, we cannot consider two contradictory possibilities at once.

%% WM is also at the heart of so-called Type 2 processes in dual process theories,
%% where reliance on WM has been proposed as a defining feature of such processes
%% \citep[see][]{Evans2013aStanovich2012}.
%% In the next section, I discuss the role of conflict
%% in dual process accounts of reasoning.



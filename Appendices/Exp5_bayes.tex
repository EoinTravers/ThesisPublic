\begin{equation}
  \begin{split}
    log(\ rt_{(p,\ d)}\ ) &=  (\Alpha + \alpha_p + \alpha_d)
    + Condition * \Beta + \epsilon \\
    \Alpha &\sim Uniform(0, 8.5) \\
    \Beta &\sim Normal(-.69, .69) \\
    \alpha_p &\sim Normal(0, \sigma_{\alpha_p}) \\
    \alpha_d &\sim Normal(0, \sigma_{\alpha_d}) \\
    \sigma^2_{\alpha_P} &\sim Cauchy(0, 10); \sigma^2_{\alpha_P} > 0 \\
    \sigma^2_{\alpha_D} &\sim Cauchy(0, 10); \sigma^2_{\alpha_D} > 0
  \end{split}
\end{equation}

This model predicts log response time ($rt$)
for participants $p$, reasoning about description $d$.
The model parameters are the overall intercept term $\Alpha$,
representing the log of the average response time when
the base rates agreed with the description,
participant $p$'s offset from this intercept, $\alpha_p$,
and description $d$'s offset, $\alpha_d$
--- both of which were normally distributed with a mean of 0,
the regression weight $\Beta$,
representing the log of the percentage increase in $rt$ when the base rate disagrees,
and the error term $\epsilon$.
$Condition$ is 0 for trials where the cues agree,
and 1 where they disagree.
A uniform prior was  set on $\Alpha$ so that
all baseline $rt$s from 1 to 5,000 msec were equally likely.
$\Beta$ was give a broad normal prior with mean 0 and SD .69,
meaning that, a priori, I was 95\% certain that
response times would be between 5 times slower and 5 times faster
than those when the cues agreed.
The variance parameters for the
by-participant and by-description differences in the intercept
had uninformative half-Cauchy distributions,
indicating that I could not specify in advance
how much variance there would be \citep{Gelman2006b}.



\subsection{Discussion}

In this experiment, participants completed 
a version of the inductive triad task using artificial categories.
Perceptual cues influenced both
participants' early cursor movements,
and their ultimate responses.
When perceptual cues and conceptual knowledge conflicted,
participants were more likely to initially move the cursor towards the foil option,
particularly if they were quick to initiate their movement,
and were also more likely to ultimately select the correct option,
regardless of initial movement.
Similarly, the time course data showed that
participants' cursor movements were driven by these perceptual cues
towards the foil option early on conflict trials,
but that this initial tendency was generally overridden.
In this sense, the current experiment
replicates the results of Experiment 1.

This experiment goes further by manipulating the kind of property
about which participants reasoned:
either specific properties, which were related to
the ontological distinction between the categories,
or uninformative, generic properties, which were not.
This manipulation had no significant effect
on participants' early movements,
and indeed in the control condition,
where participants rarely moved towards the foil option,
they appeared to play very little role at any stage.
In the conflict condition, however, where participants
initially moved towards the foil option on most trials,
I found that participants reasoning about specific, distinctive properties
were more likely to override their initial, perceptually-driven movement towards the foil
than participants reasoning about generic properties.
In other words, although reasoning about specific properties
made participants no less likely to initially rely on perceptual cues during reasoning,
it did make them more likely to reject their initial perceptually cued representation
(see Table~\ref{tbl:exp2_transitions_table}, and Figure~\ref{fig:exp2_conflict_timecourse}).

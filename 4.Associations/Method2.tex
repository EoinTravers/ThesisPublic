
\section{Experiment 4}

In Experiment 4, I attempted to replicate the findings of Experiment 3,
but without the need to discard data from
trials for which participants lacked appropriate taxonomic knowledge.
Beyond this, I wished to explore
how each participants' own beliefs about the strength of association
between various species interacted with structured knowledge during reasoning.
Therefore, in this experiment, I asked participants to rate
the association between each base species
and every response species it was paired with for the reasoning trials.
This improves on the method used in the previous study in two ways.
First, while \citet[Chapter 2]{Crisp-Bright2010} collected
ratings of the association between the base species,
the correct response species, and the strongly associated foils used in the previous experiment,
she did not collect ratings for the associations between the bases
and the foil species assumed to be weakly associated.
Second, \citet{Crisp-Bright2010} collected
association ratings from one set of participants,
and reasoning data from another.
Therefore, it was the average association rating between species
that she used as a predictor in here analyses.
Here, on the other hand, I collected each participants'
own idiosyncratic ratings of the associations between the species,
allowing a more fine-grained analysis of
how such associative knowledge influences reasoning.


\subsection{Method}

\subsubsection{Participants}

Forty four undergraduate students completed the experiment
in exchange for course credit, in a laboratory.
The experiment was programmed using the OpenSesame experiment builder
(see Chapter 2).

\subsubsection{Stimuli}

In the experimental trials, participants were asked about
biological properties, specifically cells.
Nine new stimulus sets were generated for the experimental trials,
intended to be more familiar to participants
than those used in Experiment 3.
Each new set had three foil species,
one intended to be weakly, one moderately,
and one strongly associated with the base.
Each set was presented three times, once with each foil species.
Stimuli were selected according to a number of partial pretests,
in which participants rated the strength of association between species
using the procedure from \citet[][Chapter 2]{Crisp-Bright2010}, described above.
The full set of stimuli can be found in Appendix~\ref{appendix:exp4_stimuli}.

For the filler trials, where participants were asked about diseases,
an additional fourteen stimulus sets were generated,
each containing a base species,
a correct response species likely to share a disease with the base,
and three different foil responses, one for each time the set was presented.
One possible concern about the design of Experiment 3
is that the species designated as the correct response
for each experimental stimulus set
was the correct response on every trial it featured in.
Therefore, the fourteen correct response species from the experimental trials here
were used as foil species (that is, the species that
were unlikely to share a disease with the base species)
on three different filler trials.
This meant that these species were the correct response option
in the three experimental trials in which they featured,
but also the incorrect response option in three filler trials.
The properties to be reasoned about ---
genes on the experimental trials,
and diseases on filler trials~---
were unchanged from Experiment 3.

To ensure that participants did not complete
experimental trials with the same base species in close succession,
the order of trials was randomised with constraints for each participant.
First, the experimental trials were randomly divided
into three blocks of nine trials,
with each block containing
three weak, three moderate, and three strong foils,
and one trial from each stimulus set in each.
Nine of the twenty-seven filler trials were then added to each block.
Finally, the order of trials within each block was randomised repeatedly,
until at least 5 trials separated repetitions of each base species.

\subsubsection{Procedure}

There were minimal changes to the reasoning trials from Experiment 3.
However, this experiment was conducted using the OpenSesame platform,
which allowed greater experimental control over the mouse cursor.
Therefore, instead of requiring participants
not to move the cursor during the fixation period,
the cursor's position was automatically reset
to the centre of the START button after the fixation.

After the reasoning trials, participants again completed a post-test check.
In the first part of the post-test, participants rated the strength of association
of the thirty-six base-response species pairs from the experimental trials.
These consisted of the nine base species,
each of which was paired with its correct response species
and its three foil species.
For this section, participants were presented with the following instructions,
taken directly from  \citet[][Chapter 2, p. 60]{Crisp-Bright2010}:

\begin{quote}
  [...] Please think about all kinds of possible associations, such as causal,
  functional, categorical, etcetera. Please do not think in detail about
  the mechanism by which they are related, just give your intuitive
  response. For example, if you believe that ladybirds and butterflies
  are strongly associated please give a rating closer to 9. In contrast,
  if you think cars and ladybirds are unrelated, please give a rating
  closer to 1. Please give the answer that first comes to mind, as fast
  as possible.  
\end{quote}

On each rating trial, the labelled images of each species
were shown side by side, with their positions randomised.
Participants gave their ratings by clicking on buttons
labelled 1 to 9 below the images,
with 1 subtitled ``Not associated at all'',
and 9 subtitled ``Very strongly associated''.

The second part of the post-test checked participants' structured knowledge.
Participants were presented with pairs of species,
in the same format as the association rating trials,
and gave yes or no responses by clicking the marked buttons.
There were three blocks in this part of the post-test,
and the order of trials within each block was randomised.
First, for each of the nine experimental stimulus sets,
the base species and the correct response species
belonged to the same taxonomic group.
These nine pairs of species were presented along with
an additional nine pairs that did not belong to the same group.
Participants were asked in each case if the pair shown
belonged to the same \emph{biological group}, and told
``Biological groups are the main branches
when you think about the `family tree' of the natural world'',
and that the biological groups in the experiment were
mammals, fish, reptiles, birds, and plants.

Second, for five of the experimental stimulus sets,
the base species was related to the moderately and strongly associated foils
via a food chain relationship --- one species eats the other.
These ten base-foil pairs were presented along with
ten other species pairs not related in this way.
Participants were asked if each pair shown belonged to the same \emph{food chain},
and told ``Species belong to the same food chain if one is eaten by the other
(predators and prey for animals, or a plant which is eaten by an animal)''.

Finally, for the remaining four experimental stimulus sets,
the base species was related to the moderate and strong foils
in that they shared an ecological habitat.%
\footnote{
  This was not technically true for penguins and arctic wolves,
  and penguins and polar bears, as these species live in
  opposite polar regions, but the post-test data showed that
  participants do not realise this.
}
These eight base-foil pairs were presented along with
the eight other pairs that did not share a habitat,
and participants were asked if thee pair shown ``live in the same kind of habitat''.



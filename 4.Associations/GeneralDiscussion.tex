
\section{General Discussion}

Across two experiments, this chapter investigated how
associative and structured knowledge interact during inductive reasoning,
using a task that places both forms of knowledge in conflict.
Consistent with previous work \citep[][Chapter 5]{Bright,Crisp-Bright2010,Bright2014a},
participants' inferences were influenced by both forms of knowledge.
In Experiment 3, participants overwhelmingly generalised genetic properties
among species belonging to the same taxonomic group
when the foil species was only weakly associated with the base.
When the foil was strongly associated, however,
participants selected it instead on a number of trials.
Similarly, in Experiment 4, while participants were more likely to
generalise the property to the taxonomically-related species overall,
they were less likely to do so when the foil was more strongly associated
with the base than the correct species was.

The current studies go beyond previous work
in that I recorded participants' mouse cursor movements
as they completed the reasoning tasks.
The transition probability analysis
showed the proportion of trials where the cursor was
moved initially towards each response option,
and subsequently the proportion in which
participants changed their mind to select the alternative option.
In contrast to Experiments 1 and 2,
where participants often moved towards one response and then changed their minds,
participants here generally moved towards one or other response option,
and then stayed there.
There were, however, some changes of mind;
participants who initially moved towards the foil
did sometimes change direction and select the correct species instead.
However, again in contrast to Experiments 1 and 2,
participants reversed their initial movements towards the foil
considerably less than half the time,
and additionally, these reversals made up only a small number
of the total correct responses.
The time course analyses tell a similar story.
In Experiments 1 and 2,
we saw cross-overs in the time course,
as participants were predominantly drawn towards the foil,
but then ultimately selected the correct option.
Here, on the other hand,
whenever the majority of participants were drawn towards the foil,
they majority also ended up selecting it.

Together, these patterns would suggest that
participants completing these tasks draw selectively on
\emph{either} associative knowledge \emph{or} structured knowledge.
This stands in contrast to the results of Experiments 1 and 2,
on a similar task that pitted perceptual similarity against conceptual knowledge,
which suggested that perceptual similarity was usually drawn on first,
but later overridden by conceptual knowledge.
I compare the results of these two sets of experiments
in detail in Chapter 7.

Of course, these results have implications for theories of inductive reasoning.
Broadly speaking, participants' inferences were consistent
with an account based on structured knowledge:
participants recognise that species that belong to the same taxonomic group
are more likely to share biological properties
than species that do not.
In this sense, these results are consistent with
a wealth of previous work showing that people use
information about category membership,
and between-category relationships,
when reasoning inductively
\citep{Gelman1986,Murphy2010,Murphy2004,Murphy1985,Rips1975,Osherson1990}.
However, these inferences were also influenced by associative knowledge,
a result that is difficult to account for in a theory
based on structured knowledge alone.
It supports, however, the hybrid theory of induction
proposed by
\citeauthor{Bright2014a} (\citeyear{Bright2014a}; \citealp{Crisp-Bright2010}),
which allows for both associative and structured forms of knowledge
to be used in reasoning.

Finally, the mouse cursor data collected in this experiment are somewhat equivocal.
Participants did in some cases
move initially towards the foil species cued by associative knowledge
before changing direction and selecting the correct species.
However, these reversals constituted only a minority of trajectories;
on most trials, participants moved straight to one or other species.
As noted above, these results differ from those found in Chapter 3,
where reversal trajectories were considerably more common,
and these chapters will be compared in detail in Chapter 7.















%\aside{Probably all final chapter stuff from here}
% What implications have these results for
% \citegap{Bright2014a}{'s} hybrid theory of induction?
% I have investigated the role of a number of forms of information:
% perceptual cues (similarity),
% simple knowledge (shared category membership),
% unstructured associative knowledge,
% and structured knowledge (taxonomic hierarchies).
% When perceptual cues and simple knowledge conflict,
% it appear that perceptual cues are used early on,
% with category membership activated later in reasoning.
% In contrast, when associative and structured knowledge conflict,
% participants predominantly draw on one or other kind of information
% for any given trial, rather than both.
% I believe these findings can be accounted for in two ways.
% The first possibility is that
% these two chapters have investigated qualitatively different phenomena:
% Chapter 3 concerned the interaction of perception and memory (knowledge),
% while Chapter 4 concerned different kinds of knowledge.
% If these really are distinct phenomena, then it should not be
% surprising that perception and knowledge
% (or perhaps bottom-up and top-down influences)
% should form a default-interventionist relationship,
% while different kinds of knowledge are utilised selectively
% (or perhaps activated in parallel, but with only one representation
% reaching working memory).

% The alternative possibility is that,
% rather than differing fundamentally in how they interact,
% all of these forms of information constitute
% sources which working memory can draw on
% in representing the world,
% which differ in their accessibility, their ease of use,
% and in how likely participants are to reject representations
% generated from each source once they have been formed.
% For instance, perceptually-cued representations
% may be formed quickly and easily,
% but are relatively easy to reject when inappropriate,
% as in when projecting a property which is obviously specific to animals,
% whereas associative knowledge is likely slower to retrieve,
% but more less likely to be rejected once participants have done so.
% \aside{Maybe talk about the \citet{Fisher2015} distinction between
%   perceptual and conceptual similarity here?}

% %% What the studies have in common

% %% Or a conclusion?

%%% Local Variables: ***

\section{Introduction}

In Experiment 5 (Chapter 5), I used the mouse tracking paradigm
to study the interaction of two kinds of processes in cognition:
fast, intuitive \emph{Type 1 processes},
and slower, deliberative \emph{Type 2 processes}.
In that experiment, participants were asked to respond
within 5 seconds of seeing the probe question,
and my analyses focused on the responses they gave,
their response latencies,
and whether they moved the mouse straight to their response option
or went the wrong way initially.
However, reasoning is one of the most complex,
high-level cognitive processes possible,
and naturally not all reasoning happens in less than 5 seconds.
In this chapter, I use a slightly different
kind of mouse tracking paradigm
to investigate conflict in reasoning over a considerably longer time scale.
I do this for a task
that has become perhaps the classic means of pitting
intuition against logic: the Cognitive Reflection Test \citep{Frederick2005}.
Like Chapter 5,
this chapter addresses the question of
just how Type 1 and Type 2 processes
interact in dual process accounts of cognition.

\subsection{Cognitive Reflection and Reasoning}

The Cognitive Reflection Test \citep[CRT;][]{Frederick2005}
is a brief test designed to measure
individuals’ ability to inhibit intuitive responses
in favour of reflective and deliberative reasoning.
In the bat-and-ball problem, one of the best-know CRT items,
participants are asked:

\begin{quote}
  ``A bat and a ball together cost £1.10.\\
    A bat costs £1 more than a ball.\\
    How much does a ball cost?''
\end{quote}



The appealing but incorrect response, to say “10p”,
is believed to be generated effortlessly and automatically.
Arriving at the correct response of “5p” may require that
this intuitive response is inhibited in favour of
the result of sustained, effortful deliberation.

The CRT has become a widely-used measure of individual differences in cognition.
Higher CRT scores predict better performance on various cognitive tasks,
including reduced framing effects,
less discounting of delayed rewards
\citep{Cokely2009,Frederick2005}
and probability matching \citep{Koehler2009},
resistance to the illusion of explanatory depth \citep{Fernbach2013}
and conjunction fallacies \citep{Oechssler2009},
greater metacognitive awareness \citep{Mata2013a},
and less endorsement of supernatural beliefs
\citep{Pennycook2012,Shenhav2012},
less endorsement of vacuous statements as profound \citep{Pennycook2015b},
as well as performance on various tasks that pit normative responding against intuition
\citep{Toplak2011}
Scores on the CRT correlate with measures of IQ and personality characteristics,
and usually predict performance even when these are controlled for
\citep{Toplak2011}.


\subsection{Dual Process Accounts}

The CRT is widely seen as a an archetypal application of
dual process theories of cognition \citep{Frederick2005,Kahneman2005,DeNeys2013a}.
Consistent with this, performance on the CRT is related to
performance on a number of other traditional dual process tasks \citep{Toplak2011},
and to dispositional factors related to willingness to engage in analytic thinking
\citep{Campitelli2010a,Campitelli2013,Bockenholt2012}.

However, dual process theories differ in their account of CRT performance.
As discussed in Chapter 1, there are a number of ways in which
Type 1 and Type 2 processes can interact during reasoning.
As has been my focus throughout this thesis,
I will concentrate on the predictions made by each account
with regard to when conflict occurs during reasoning.
Although I am interested in reasoning in general,
I discuss these accounts in terms of
the specific points at which they predict
conflict should occur during the CRT.
According to a selective dual process account \citep[e.g.][]{Klaczynski2004},
the CRT, like all such tasks, should not evoke any conflict.
Instead, participants should selectively draw on heuristic Type 1 processes,
which usually produce a response of ``10p'' on the bat-and-ball problem,
or draw on Type 2 processes to apply the simple mathematical rules
needed to reach a correct response of ``5p''.
On the other hand, Type 1 processes are always automatically activated
in default-interventionist models \citep{Evans2006,Kahneman2011,Kahneman2005}.
Therefore, such accounts would predict that
heuristic responses (``10p'') come to mind automatically for most participants,
and that giving the correct response requires that this heuristic response is inhibited,
and that Type 2 processes are engaged to derive the correct response.
Such a default-interventionist account is assumed, for instance,
in \citegap{Frederick2005}{'s} paper introducing the CRT,
and \citegap{Toplak2011}{'s} work outlining the relationship between
the task and other dual process problems.

A third option \citep{Sloman2014,Sloman1996}
is that both Type 1 and Type 2 processes are activated simultaneously,
and that they compete for control of behaviour.
In common with default-interventionist models,
these accounts predict that Type 1 intuitive responses
must be inhibited in order to reason correctly.
Uniquely though, parallel models would also predict Type 2 processes
should attempt to signal the correct response,
even when failing to overrule the output of Type 1 processes.

The intuitive logic model \citep{DeNeys2012,DeNeys2014a}
has also been applied to performance on the CRT \citep{DeNeys2013a}.
This model modifies the traditional default-interventionist model
to account for many findings which indicate that
when participants provide biased, heuristic responses,
they are often implicitly aware of some conflict between
their responses and the normative standard.
According to this model, Type 1 processes are sensitive to normative principles,
such as logical principles in syllogistic reasoning tasks,
or mathematical rules on the bat-and-ball problem.
As a result, they implicitly signal a conflict when
the incorrect heuristic response is given.
However, because the heuristic response is usually prepotent,
participants often fail to inhibit it,
even when they do detect that it conflicts with normative principles.
It is unclear at present, however, how this conflict is actually detected.
One possibility is that Type 1 processes simultaneously produce
both heuristic and correct responses,
and it is the conflict between these two partially active beliefs
which is detected directly.
In Chapter 1, it is this proposition that I refer to
as the intuitive logic theory.
Alternatively, the process may be more subtle,
with Type 1 processes not generating a fully-formed correct response,
but rather detecting, through some other means,
that the heuristic response is questionable.
I refer to this proposition in Chapter 1
as the dual process conflict monitoring theory.
Clearly, these two possibilities make different predictions
about conflict between competing response options during reasoning.
In the former case, the intuitive logic model would,
like a parallel-competitive account, predict that
because both responses are partially cued,
participants should be drawn towards giving the correct response,
even when they ultimately give the heuristic one.
In the latter case, if Type 1 processes can signal conflict
without actually generating the correct response,
participants may experience conflict and uncertainty,
but not be actually drawn towards the correct response when giving the heuristic one.
For the purposes of this experiment,
the same predictions are made by the slightly different
intuitive logic account offered by \citet{Handley2015}.
For simplicity, I will refer to \citegap{DeNeys2012}{'}
account throughout, with the understanding that the same points generally apply
to other intuitive logic accounts.

As discussed in previous chapters, evidence of
the implicit conflict detection predicted by the intuitive logic model
comes from a range of experimental paradigms
\citep[see also][for a review]{DeNeys2012}.
Typically, these studies compare conflict problems,
where the intuitive, heuristic response is incorrect,
to analogous no-conflict versions,
where both heuristics and normative principles cue the same response.
Type 1 processes cue both the heuristic response on conflict problems
and the correct response on no-conflict problems.
If participants detect the conflict between
normative principles and their heuristic responses,
they should show greater evidence of conflict on these problems,
compared to the no-conflict problems.
Such conflict has been measured using confidence ratings
\citep{DeNeys2011b},
response times \citep{DeNeys2008},
neuroimaging \citep{DeNeys2008a},
and galvanic skin response \citep{DeNeys2010},
among other measures.

Two studies, however, directly test the intuitive logic model against the CRT.
\citet{DeNeys2013a} showed that heuristic responses on conflict problems
were given with less confidence than correct responses on no-conflict versions of the same problems.
\citet{Gangemi2015} report similar effects,
asking participants to fill out a brief questionnaire
measuring their ``feeling of error'' after answering either
the original bat-and-ball problem or a no-conflict control version,
both when participants were asked to generate their responses,
and when asked to choose between the heuristic and correct responses.
These findings all suggest that participants are to some extent
aware of the inadequacy of their heuristic responses. 


In this chapter I use a novel version of the mouse tracking paradigm
to explore this topic.
Participants completed a computer-based multiple-choice version of the CRT,
including both conflict problems,
where the intuitively appealing heuristic response was incorrect,
and no-conflict versions of the same problems,
where the appealing response was the correct one.
There  were four response options for each problem,
located in each corner of the display.
While participants decided on their response,
their mouse cursor movements were recorded
as they moved the cursor around the screen.
Unlike in previous mouse tracking studies,
participants were not placed under time pressure,
and their mouse movements over the first 60 seconds of each trial were analysed.
Rather than moving the cursor directly from its starting point
to a response option and clicking on it,
participants typically moved the cursor
around the screen a number of times on each trial,
passing close to multiple response options.
As mentioned in Chapter 1,
mouse movements over such long time scales
have been used in the past
as part of research on human-computer interaction,
typically involving users interacting with web pages,
such as web search results pages
\citep[e.g.][]{Chen2001,Rodden2008,Huang2011}
This allowed me to use the patterns in these cursor movements
to test predictions derived from the various forms of dual process theory.

According to a selective dual process account,
participants should move to the response option they eventually chose
without showing any particular attraction towards
any of the other options along the way.
According to a traditional default-interventionist account,
on conflict trials participants should initially be drawn
towards the heuristic response option,
but in some cases inhibit this response and instead select the correct option.
A parallel-competitive account would likewise predict that participants
are drawn towards the heuristic option on trials where they
eventually give the correct response,
but would also predict the reverse ---
because Type 2 processes are activated in parallel with Type 1,
participants should be in some cases drawn towards
the correct option even on trials where they
end up giving the heuristic response.
The predictions of the intuitive logic model
depend on the nature of the conflict detection process.
If participants detect conflict because
both responses are simultaneously generated by Type 1 processes,
then the intuitive logic model, like the parallel-competitive model,
would predict conflict in both directions.
Alternatively, if the conflict detection process is more subtle
(c.f. the dual process conflict monitoring theory, Chapter 1)
then like the classic default-interventionist account
it would predict that participants should be drawn to the heuristic option
when selecting the correct one, but not the other way around.


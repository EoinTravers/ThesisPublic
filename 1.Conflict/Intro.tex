
\section{Introduction}\label{sec:chapter1-intro}

Reasoning is not a simple task.
Below the surface of everyday judgement, decision making, and high-level thinking,
there are a range of underlying processes,
many of which we are only beginning to understand.
Unsurprisingly, amid the constant interaction of these processes,
there are many points at which conflict arises in reasoning,
and in cognition more broadly.
Much previous work
\citep{Botvinick2004a,Veen2006,Spivey2007}
has addressed the role of conflict
within and between individual low-level cognitive processes.
In this thesis, I focus on conflict at a higher-level:
conflict in reasoning.

I focus on two major points at which conflict
is thought to arise during reasoning.
First, many forms of reasoning,
and inductive reasoning in particular,
must draw on various sources of information,
including new information, and prior knowledge.
Conflict can arise when these different sources of information
suggest different inferences or actions.
Second, many problems are thought to be attempted using either
fast, automatic, effortless \emph{Type 1} processes,
or using slower, deliberate, effortful \emph{Type 2} processes,
and an extensive \emph{dual processes} tradition
focuses on the characteristics of these processes and how they interact.
Here, conflict arises when multiple responses are cued simultaneously.
This thesis presents four experimental chapters,
two studying conflict between mental representations in induction,
and two studying conflict between Type 1 and Type 2 processes in reasoning.

In most experimental studies of reasoning,
researchers record and analyse only participants' responses,
or their responses and their responses times.
This approach has produced considerable insights,
but it also has limitations.
In particular, it is difficult to infer
what processes actually drive reasoning
by analysing only responses which are recorded
after these processes have ended.
Therefore, to study  conflict in reasoning,
I use a mouse tracking paradigm.
This technique, popular in the studies of conflict in simpler cognition,
involves recording the position of the computer mouse cursor
as participants decide between alternative response options.
By analysing these cursor trajectories,
it is possible to infer to what extent participants
are drawn towards each option, over time.
In particular, this method allows me to infer two things in this thesis:
what responses were participants drawn towards giving
\emph{before} giving their ultimate response,
and at what points in time were they drawn towards each response.

The remainder of this first chapter is organised as follows.
First, I discuss previous work on
the different kinds of information
that can be drawn on in inductive reasoning,
and the possibility of conflict between them.
Next, I introduce dual process theories of cognition,
and review recent work on the nature of conflict within such theories.
Lastly, I present an overview of methods used to study conflict in cognition,
and introduce the mouse tracking paradigm used throughout this thesis.


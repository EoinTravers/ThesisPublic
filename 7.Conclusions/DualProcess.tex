
\section{Dual Processes and Reasoning}\label{sec:ch7-dualproc}

In Experiments 5 and 6 (Chapters 5 and 6),
I tested the predictions of dual process theories of reasoning.
At the outset, I introduced a number of types of dual process architecture
outlined by \citet{Evans2007a}:
selective/pre-emptive conflict resolution models,
where reasoners selectively draw on Type 1 \emph{or} Type 2 processes each time;
corrective/default-interventionist models,
where Type 1 processes are activated automatically,
and optionally overridden by Type 2 processes;
and parallel-competitive models,
where both types of process operate in parallel,
and compete for control of behaviour.
I also discussed
what I referred to as
the dual process conflict monitoring account
\citep[e.g.][]{DeNeys2008,DeNeys2008a},
which proposes that participants implicitly detect
that their heuristic reasoning conflicts with normative standards,
and the more recent \emph{intuitive logic} accounts of reasoning
\citep{DeNeys2012,DeNeys2014a,Handley2015},
which propose that this conflict detection happens because
Type 1 processes can simultaneously cue both
biased heuristic responses and normatively correct ones.

These accounts differ in when they predict conflict should occur,
and mouse tracking provides a means of detecting this conflict when it does take place.
Pre-emptive conflict resolution models predict that Type 1 and 2 processes never conflict,
as we selectively activate only one or other kind of process on each task.
Default-interventionist models predict conflict
only when we activate Type 2 processes to override Type 1 responses.
Parallel-competitive models, however, predict conflict to be more common,
as both types of process usually compete to control behaviour.
Like parallel-competitive models, intuitive logic accounts
would predict that we are often conflicted during reasoning,
but while the former claim this is because Type 1 and Type 2 process cue conflicting responses,
the latter claim this occurs because Type 1 processes simultaneously cue multiple responses,
often including both biased heuristic responses and normative correct ones.

Experiments 5 and 6 both tested these various predictions:
Experiment 5 using the base rate neglect paradigm,
where participants responded within five seconds,
and Experiment 6 using the four-alternative Cognitive Reflection Test \citep[CRT;][]{Frederick2005},
where participants were not placed under time pressure.
In both cases, I found results strongly in favour of a default-interventionist model.
First, in the base rate neglect task,
Type 1 processes are traditionally thought to underlie description-driven responses,
and Type 2 processes to underlie base rate-driven responses
\citep{Barbey2007,Kahneman2005,Kahneman2002}.
Participants' early, initial movements were sensitive
to manipulations of the description,
but not to manipulations of the base rate,
as they moved towards the description-cued option
\tildetext66\% of the time, regardless of the base rate.
It was only later in the reasoning process (after 750 msec or so)
that base rates began to influence participants' movements,
and ultimately, some of their responses.
Therefore, it appears that descriptions were processed automatically, by default,
and only in some cases did participants
attend to the base rates and override their default responses.
Also consistent with a dual process interpretation,
when the two cues conflicted participants
overwhelmingly gave the description-cued response,
and were less conflicted while doing so.

In the CRT, similarly,
Type 1 processes are traditionally held \citep{Frederick2005,Kahneman2005}
to drive heuristic responses
(e.g. ``10p'' on the bat-and-ball problem)
whereas Type 2 processes drive correct responses (e.g. ``5p'').
Consistent with this, participants predominantly gave
the heuristic response on conflict problems,
and were faster to do so, showed less signs of conflict,
and were also faster to approach this heuristic response before clicking it.
Analysis of participants' cursor movements also showed that
participants giving the correct response
were more likely to hover in the region of the heuristic option
than any of the other foil options before doing so.
This would indicate that these heuristic responses
must be inhibited before Type 2 processes can produce the correct response.

The dual process conflict monitoring theory \citep[e.g.][]{DeNeys2008,DeNeys2008a},
intuitive logic accounts \citep{DeNeys2012,DeNeys2014a,Handley2015},
and the parallel-competitive dual process theory \citep{Sloman2014,Sloman1996}
make additional predictions about when conflict should occur.
According to the conflict monitoring theory, participants responding heuristically
should show evidence of conflict on problems where
their response is not also the normatively correct one,
compared to problems where it is
(in other words, they show signs of conflict when
their heuristic response is wrong, but not when it is right).
The intuitive logic and parallel-competitive accounts go further,
and propose that this conflict occurs because both
heuristic and correct responses are being cued simultaneously.
The former proposes that Type 1 processes cue both responses
but that the heuristic response, being prepotent, is the one normally given.
The latter proposes that Type 2 processes cue the correct response
at the same time as Type 1 processes cue the heuristic one.

In the two-alternative base rate neglect task,
I tested these predictions by looking to problems where
participants give the description-cued response,
while the base rate either agreed with the description,
was uninformative, or disagreed.
Previous studies \citep[i.e.][]{DeNeys2008a,DeNeys2008,Pennycook2012a,Pennycook2014}
have found that participants are slower to respond
when the description disagrees with the base rate.
Here, I did not find such an effect,
in response times or any other measure ---
although there was not strong evidence in favour of a null effect either.
It appears that participants either
gave the base rate-cued response, or (largely) ignored the base rates altogether.
Therefore, while these results do not support the intuitive logic theory,
they do not falsify it either.

It should be noted, however,
that most studies showing an intuitive logic effect in this paradigm
involved participants responding much more slowly than was the case here.
Therefore, it may be that participants rarely process base rates
when required to respond in less than six seconds (see the Discussion of Chapter 5).
I asked participants to respond quickly in this experiment,
as in Experiments 3 and 4, in order to ensure that
they were still making their decisions while moving the mouse cursor
so that the cursor trajectories would reflect the reasoning process.
In Experiment 6 however,
due to the fortuitous fact that people will
spontaneously move the cursor while inspecting and thinking about
options located around the screen,
I was able to let participants complete the CRT in their own time.

The design of this experiment allowed me to disentangle
the predictions of the dual process conflict monitoring theory
--- heuristic responses to be slower when
they are incorrect than when they are correct ---
from those of the stronger intuitive logic theory,
and of a parallel-competitive dual process model
--- attraction towards the correct option
before selecting the incorrect heuristic option on conflict problems.
However, I not only failed to confirm these predictions,
but in both cases in fact found significant effects in the opposite directions.
Participants were slower to approach the heuristically-cued correct option
on no-conflict problems where it was selected
than to approach the heuristically-cued incorrect option
on conflict problems where it was selected.
Likewise, participants giving the correct response on conflict problems
actually spent \emph{less} time in the region of the correct option
than the other foil options.
However, these results were not predicted by any of the theories considered,
and so were unexpected.
Despite this, it seems that neither the imposition of time pressure
nor a lack of statistical power
can explain the absence of effects
consistent with the intuitive logic theory here.

%% \aside{Link sentence?}

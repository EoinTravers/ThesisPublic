\subsection{Logic and Biases}
\label{subsec:chapter1-dual-process-logic-and-biases}

A common thread through many dual process theories is that rationality --
making optimal decisions, or conforming to logical and mathematical rules --
requires the operation of Type 2 processes,
while Type 1 processes often lead to errors, biases, and irrationality.
In the field of judgement and decision making (JDM),
dual process theories are strongly linked
to the earlier Heuristics and Biases research tradition
\citep{Gilovich2002,Tversky1974,Kahneman1982}. %Others to cite?
In this, a number of heuristics, or mental shortcuts
that people rely instead of engaging in more effortful decision making,
lead to systematic biases in situations in which
the heuristic does not provide a useful solution
to the problem at hand.
Kahneman (\citeyear{Kahneman2011}, \citealp{Kahneman2002})
explains these heuristics as cases of \emph{attribute substitution}:
participants answer an easier question then the one asked.
The number of heuristics documented in this literature is considerable
\citep[see][for reviews]{Gilovich2002}.
Examples include estimating the probability of an event
based on how easily it comes to mind
\citep[the availability heuristic;][]{Tversky1973},
estimating the likelihood of something belonging to a given category
according to how representative it is of that category
\citep[the representativeness heuristic;][]{Kahneman1973,Kahneman1972},
and biasing numerical judgements towards
recently seen irrelevant numbers 
\citep[anchoring and adjustment;][]{Tversky1974}.
Use of these heuristics leads to systematic biases
in many situations.

More recent treatments of the heuristics and biases literature
\citep{Kahneman2005,Kahneman2002,Kahneman2011}
have explicitly described heuristics as the result of Type 1 processes,
while the \emph{reflective} thinking
needed to resist these heuristic responses \citep{Frederick2005}
is the consequence of Type 2 processes.
Similarly, in dual process theories of reasoning
\aside{Are Reasoning and JDM really two different fields in this regard?}
Type 1 processes are thought to lead to systematic departures from
the relevant normative standards.
For instance, Type 1 processes are thought to make participants
select cards explicitly mentioned in the question
in the Wason selection task \citep{Wason1975},
rely on the belivability of conclusions, rather than logical validity,
in evaluating logical syllogisms
\citep[belief bias;][]{Evans1983,Klauer2000},
and to treat fractions with large numerators (30/100)
as larger than those with small ones
\citep[3/10, ratio bias;][]{Kirkpatrick1992}, among other biases.
%% \aside{Or is ``denominator neglect''?}
Correspondingly, Type 2 processes are thought to be required
to follow the normative rules of reasoning,
be they classical logic \citep{Rips1994,Braine1998}
or probability theory \citep{Chater2010,Chater2006}.
Social psychology, likewise,
attributes to Type 1 processes such phenomena as
stereotyping
\citep{Mason2006,Greenwald1995},
and persuasion through irrelevant cues
\citep{Petty1996,Chaiken1987,Petty1986,Chaiken1980}.


% Intuitive logic
This distinction has been challenged, however, on a number of grounds.
Most relevant to the current discussion is
the recent \emph{intuitive logic} theory
\citep{DeNeys2012, DeNeys2014a},
which suggests that we may be able to effortlessly
and spontaneously reason logically.
This account has important implications for the study
of conflict in reasoning,
and so I will take some time to introduce it here.

As noted above, classical dual process theories
hold that Type 1 processes lead to biased reasoning,
while Type 2 processes are often necessary for rational thinking.
From this perspective \citep[i.e.][]{Evans2006},
human reasoning often departs from normative standards of rationality
because we are often content to rely on the output of Type 1 processes.
Although the purpose of Type 2 processes in this framework
is to monitor and correct these outputs if necessary,
this monitoring is thought to be quite lapse ---
in other words, we are thought to be \emph{cognitive misers} \citep{Fiske1991}.
Crucially, because they are usually generated easily and fluently
\citep{Thompson2013,Thompson2012a},
and not subjected to proper inspection,
participants are often thought to be
blissfully unaware that their Type 1 responses may be biased
\citep[see][]{Kahneman2005,Kahneman2011}.

Subtle experiments, however,
using tasks where participants predominantly give
the incorrect, heuristic response,
and comparing performance on these \emph{conflict} tasks
to \emph{no-conflict} equivalents where
the heuristic response is the correct one, cast doubt on this idea.
In these experiments, participants rarely
explicitly report that they realise their heuristic responses
are suspect, or run counter to normative principles \citep{DeNeys2008}.
However, more subtle measure show evidence that
they are sensitive to the conflict between
their heuristic responses and normative principles;
for instance, on conflict trials participants
are slower to respond \citep{DeNeys2008a},
are less confident in their responses \citep{DeNeys2011b, DeNeys2013a},
sweat more \citep{DeNeys2010},
show greater activation in the anterior cingulate cortex,
a neural region implicated in the detection of conflict \citep{DeNeys2008a},
and even report ``liking'' logically valid syllogisms
more than equivalent invalid ones \citep{Morsanyi2012}.
For clarity, I will refer to De Neys' \citep[e.g.][]{DeNeys2008a} proposal
that reasoners who give heuristic responses
detect some conflict while doing so
as the \emph{dual process conflict monitoring theory}.
This accounts draws on theories of conflict monitoring
in simpler cognitive domains \citep{Botvinick2001},
discussed below.

In more recent work \citep[i.e.][]{DeNeys2012, DeNeys2014a},
De Neys has proposed a stronger interpretation of these results:
Type 1 processes can simultaneously cue
both the sometimes incorrect heuristic response
and the logically correct response.
Clearly, this is a stronger claim than that made by
the dual process conflict monitoring theory,
and I will reserve the name \emph{intuitive logic} theory for this account.
According to this stronger account,
participants predominantly give the heuristic response
because it is cued more strongly than the correct one (i.e. it is \emph{prepotent}).
Type 2 processes, however, are activated when
participants detect a conflict between
multiple responses generated by Type 1 processes,
and must inhibit the prepotent heuristic response
in order for participants' to produce the correct response.
However, according to \citet{DeNeys2013},
while participants will usually detect this conflict,
and attempt to engage Type 2 processes,
they are often unable to inhibit the heuristic response,
leading to biased reasoning.

%% Unlike the Stroop task, however,
%% it is less clear if people are aware that 
%% their heuristic responses are inappropriate when reasoning.
%% De Neys (\citeyear{DeNeys2008a}; \citealp{DeNeys2012,Franssens2009})
%% cites the Conflict Monitoring Theory \citep{Botvinick2004}, above,
%% which predicts that if both heuristic and logical responses
%% are cued by Type 1 processes
%% people should be to some degree aware of this conflict,
%% even if they fail to inhibit their prepotent responses.
%% In support of this, \citet{DeNeys2008a} show that
%% participants solving problems in which the
%% heuristic response conflicts with logical norms
%% show increased ACC activation,
%% corresponding to conflict detection.
%% At the same time, only those who reasoned logically,
%% and therefore overcame the prepotent heuristic responses,
%% showed increased dlPFC activation,
%% reflecting inhibitory processes.
%% In this same research programme,
%% evidence suggesting that participants experience conflict
%% when heuristics and logical principles disagree
%% has come from analysis of response latencies \citep{DeNeys2008}
%% confidence ratings \citep{DeNeys2013a},
%% galvanic skin conductance \citep{DeNeys2010},
%% visual fixations \citep{DeNeys2008},
%% and even analyses of how much participants
%%  ``like'' certain syllogisms \citep{Morsanyi2012}.

\citet{Handley2015} offer a tentative dual process theory
that is in many ways similar to \citegap{DeNeys2012}{'} intuitive logic account.
They propose that many kinds of information,
traditionally thought to be processed exclusively by
either Type 1 or Type 2 processes, can in fact be processed by either.
For instance, in the phenomena known as belief bias \citep{Evans1983},
%% \aside{Make sure I've not introduced this already}
participants asked to evaluate the logical validity of syllogisms
--- to decide, based on the structure of the argument,
if the conclusion necessarily follows from the premises ---
were more likely to say a syllogism was valid if
its conclusion was believable,
and invalid if its conclusion was unbelievable.
Therefore, a Type 2 process (evaluating logical validity)
is interfered with by a Type 1 process (retrieving beliefs).
\citet{Handley2011} both asked participants
to evaluate the logical validity of syllogisms
while conclusion belivability was manipulated,
and to evaluate the belivability of their conclusions
while validity was manipulated.
They found conflict in both directions:
belivability affected judgements of validity,
and validity affected judgements of belivability,
with participants making more mistakes, and responding more slowly
when the cues conflicted.
\citet{Pennycook2014} report a similar finding
in the base rate neglect paradigm (see Chapter 5),
as statistical information traditionally thought
to be processed by Type 2 processes only
interfered with judgements based on stereotypes,
thought to be the produced of Type 1 processes.

%% \aside{
%%   I could also talk about some of the Thompson and Handley work
%%   suggesting that beliefs can be slow, and logic fast,
%%   but that might be a bit tangential}
%% \aside{
%%   There's more recent work showing more or less the same thing
%%   (according to people at LRW this year), which I need to find and incorporate here
%% }

While they differ in some regards,
the accounts put forward by \citet{DeNeys2012} and \citet{Handley2015}
both depart from classical dual process theories in that
they propose that Type 1 processes can produce responses
that are consistent with normative principles.
This differences becomes particularly important in the next section,
where I discuss how Type 1 and Type 2 processes interact, and conflict,
during reasoning.

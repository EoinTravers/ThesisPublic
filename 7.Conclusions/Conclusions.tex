
%% can you find another title for this sub-section. I expected it to be
%% the end of the chapter. Actually, I'm a bit worried about the
%% structure here. Would it be better to save these conclusions for a
%% Conclusions and Final Statement section? You could pivot at this point
%% saying something liike The previous section concerned theoretical
%% bases for predicting different kinds of mouse movement patterns. In
%% this section I want to reflect on what can be learned from my
%% programme of experiments about how the method may be applied to high
%% level cognition such as reasoning. Then, next section, with this in
%% mind,, we can think about potential future uses of the method to study
%% reasoning. And then, at the conclusion of this thesis it is usefully
%% to briefly summarise what has been shown and....

%% \section{Conclusions}\label{sec:ch7-conclusions}


\section{Conclusions}

To recapitulate, in this thesis I used the mouse tracking paradigm
to study conflict in reasoning.
Using it to study inductive reasoning, I have shown that
a) perceptual cues have a major influence on
the early stages of inductive reasoning,
with conceptual knowledge being retrieved and used later;
b) perceptually-driven movements are more likely to be
overridden by conceptual knowledge when participants reason about
properties conceptually related to that knowledge;
c) associative and structured knowledge also conflict during reasoning, but
d) compared to perceptual cues, associative knowledge
is less likely to drive early movements,
but also less likely to be overridden when it does drive them,
and I propose that
e) the latter effect is the result of conflict between
competing mental representations attempting to enter working memory,
while the former may or may not be mediated by working memory at all.

Likewise, studying reasoning from a dual process perspective, I have shown that
f) participants reasoning based on stereotypical descriptions
experience less conflict than those reasoning based on statistical base rates
when the two cues conflict;
g) descriptions influence the reasoning process both early on and throughout,
while base rates are only influential later in reasoning;
h) participants responding according to the base rate
are affected by manipulations of the description,
showing signs of conflict when the description disagrees with the base rate;
i) participants responding according to the description
were minimally affected by manipulations of the base rate ---
participants either ignored the base rate altogether,
or attended to it, and gave the response it cued;
j) heuristic responses on the CRT are approached more quickly than correct ones;
k) heuristic responses that are also correct
are not approached or chosen any more quickly than those that are incorrect;
l) participants selecting the correct response option on the CRT
are drawn towards the heuristic option before doing so; but, finally,
m) participants selecting the heuristic option
are not drawn towards the correct one.


\subsection{Closing Statement}

This is a thesis on conflict in reasoning.
Using the mouse tracking paradigm as a continuous measure
of participants' attraction towards competing response options during reasoning,
I have demonstrated where this conflict does, and does not occur.
Beyond this, I have shown something of the nature of this conflict;
it is a discrete, all-or-nothing effect,
as a number of distinct cognitive processes ---
including both associative, autonomous Type 1 processes,
and Type 2 processes dependent on working memory ---
vie for control over cognition, and over the motor system.
Outside of the specific tasks I have studied, however,
hopefully this thesis will provide a useful set of
practical and conceptual tools for making sense of
the processes that underlie human reasoning.
Ultimately, while I hope I have helped to answer some questions here,
my real hope for this thesis is that
it encourages readers
to ask, and to answer, new questions
about how we reason about the world.

%% I like this last sentence, but I'm always dubious about the "as Xs"
%% construction e.g. "as humans". Might it be better to write "that it
%% encourages readers to ask, and to answer..."

%% Reading this final statement I am convinced that you could place
%% yoour earlier conclusions section at its beginning without very
%% much re-writing.
